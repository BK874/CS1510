\documentclass[letterpaper,notitlepage,twoside]{article}

% Basic imports, increase margins...
\usepackage[margin=0.75in]{geometry}
% Finite State Machine stuff

% Format tables nicely
\usepackage[latin1]{inputenc}
\usepackage{array}

\usepackage{amsfonts} 
\usepackage{amssymb}
\usepackage{amsmath,amsthm}

\renewcommand{\implies}{\Rightarrow} % redefine command "implies"  
\renewcommand{\iff}{\Leftrightarrow} % double arrow
\newcommand{\maps}{\rightarrow} % define command "map" 
\newcommand{\union}{\cup}
\newcommand{\intersect}{\cap}
\newcommand{\N}{\mathbb{N}} % natural number 
\newcommand{\Q}{\mathbb{Q}} % rational number 
\newcommand{\R}{\mathbb{R}} % real number 
\newcommand{\Z}{\mathbb{Z}} % integers 
\newcommand\tab[1][1cm]{\hspace*{#1}} %\tab command

% Add more packages that you use here...

\begin{document}
\title{Homework 19}
\author{Brian Knotten, Brett Schreiber, Brian Falkenstein}
\maketitle
\section*{1}
Here, we will reduce standard matrix multiplication to multiplication of two lower triangular matrices. Assume we have an algorithm $multTri(A, B)$ that solves the multiplication of two lower triangular matrices in $O(n^2)$ time. Given as input to standard matrix multiplication two matrices $A$ and $B$, both of which are $n\times n$, we can construct two new matrices that will be passed into $multTri(A', B')$ by doing the following: 
\[
A' = 
\begin{bmatrix}
    0 & 0 & 0 \\
    0 & 0 & 0 \\
    0 & A & 0 
\end{bmatrix}
\]
\[
B' = 
\begin{bmatrix}
    0 & 0 & 0 \\
    B & 0 & 0 \\
    0 & 0 & 0 
\end{bmatrix}
\]
Where $0$ represents an $n \times n$ matrix of all zeros. Both of these matrices will be $3n\times 3n$. Both transformations take $n^2$ time, as it would take constant time to construct a $3n\times 3n$ matrix of zeros, then $n^2$ time to copy over the elements of $A$ and $B$ into the correct places in $A'$ and $B'$. Multiplying $A'$ and $B'$ gives us the $3n \times 3n$ array: \\
\[
A' \times B' = 
\begin{bmatrix}
    0 & 0 & 0 \\
    0 & 0 & 0 \\
    AB & 0 & 0 
\end{bmatrix}
\]
We can then output the bottom left $n\times n$ elements in the matrix, which takes $n^2$ time. These elements are exactly the values of the elements of $A\times B$. Thus, we have converted the input from matrix multiplication to the input for $multTri(A,B)$ in $n^2$ time, and converted the output from $multTri(A,B)$ to the output for standard matrix multiplication in $n^2$ time, so if an $O(n^2)$ algorithm exists for the multiplication of two lower triangular matrices, so does one exist for standard matrix multiplication. 
\section*{2}
We will reduce standard matrix multiplication to the inversion of a nonsingular matrix. Assume we have an algorithm $invertMatr(C)$ that solves the inversion of a nonsingular matrix in $O(n^k)$, $k \geq 2$ time. Given two $n \times n$ matrices $A$ and $B$ as input to matrix multiplication, we can construct a new matrix to be passed into $invertMatr(C)$ as follows:
\[C = \begin{bmatrix} I & A & 0 \\ 0 & I & B \\ 0 & 0 & I \end{bmatrix} \]
Where $0$ represents an $n \times n$ matrix of all zeros and $I$ represents the Identity matrix with ones on the main diagonal and zeros everywhere else. $C$ will be $3n \times 3n$. The construction of $C$ will take approximately $n^2$ time, as it takes constant time to construct a $3n \times 3n$ identity matrix, then $n^2$ time to copy the elements of $A$ and $B$ over to the correct places in $C$. Inverting $C$ gives us the $3n \times 3n$ matrix: \\
\[C^{-1} = \begin{bmatrix} I & -A & AB \\ 0 & I & -B \\ 0 & 0 & I \end{bmatrix} \]
We can output the top right $n \times n$ elements in the matrix $C^{-1}$, which are exactly the values of the elements in $A \times B$, in $n^2$ time. Thus, we have converted the input from the matrix multiplication of two $n \times n$ matrices to the input for $invertMatr(C)$ in $n^2$ time and converted the output from $invertMatr(C)$ to the output from matrix multiplication in $n^2$ time. Therefore, if an $O(n^k)$, $k \geq 2$ algorithm exists for the inversion of a nonsingular matrix, then an $O(n^k)$ algorithm also exists for the matrix multiplication of two $n \times n$ matrices. 
\end{document}
