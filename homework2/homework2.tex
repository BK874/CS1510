\documentclass[11pt, oneside]{article}   	
\usepackage{graphicx}				
										
\usepackage{amssymb}



\title{CS 1510 Homework 1}
\author{Brian Falkenstein, Brian Knotten, Brett Schreiber}

\begin{document}
\maketitle

{\noindent \large Problem 2 B \par}
Assume the algorithm given, A, does not solve the interval coloring problem. Let s be the maximum number of intervals that overlap at any point, and thus, the optimal output (minimum number of rooms after assigning all intervals). Then, A will output a value larger than s for some input, and thus, A does not solve the problem. We can conclude that in the case that A's output is greater than s, this means that A assigned a class to a new room when another room already had a space for it, as this would be the only case where A's output could be greater than s. Because, in this algorithm, classes already placed are never altered / assigned to another room, this is a contradiction. The algorithm checks the set R of rooms that already have a class scheduled in it, and places the class in the first room it finds that does not cause an overlap. Thus, the output of A cannot be larger than s. 
~\\ \par
\end{document}