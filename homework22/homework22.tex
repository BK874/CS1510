\documentclass[letterpaper,notitlepage,twoside]{article}

% Basic imports, increase margins...
\usepackage[margin=0.75in]{geometry}
% Finite State Machine stuff

% Format tables nicely
\usepackage[latin1]{inputenc}
\usepackage{array}

\usepackage{amsfonts}
\usepackage{amssymb}
\usepackage{amsmath,amsthm}

\renewcommand{\implies}{\Rightarrow} % redefine command "implies"
\renewcommand{\iff}{\Leftrightarrow} % double arrow
\newcommand{\maps}{\rightarrow} % define command "map"
\newcommand{\union}{\cup}
\newcommand{\intersect}{\cap}
\newcommand{\N}{\mathbb{N}} % natural number
\newcommand{\Q}{\mathbb{Q}} % rational number
\newcommand{\R}{\mathbb{R}} % real number
\newcommand{\Z}{\mathbb{Z}} % integers
\newcommand\tab[1][1cm]{\hspace*{#1}} %\tab command

% Add more packages that you use here...

\begin{document}
\title{Homework 22}
\author{Brian Knotten, Brett Schreiber, Brian Falkenstein}
\maketitle
\section*{8}
\subsection*{HamiltonianCycle $\leq_p$ DoubleFixedHamiltonianPath}
HamiltonianCycleAlgorithm($G$):\\
\tab return $\bigvee_{v \in G}$ DoubleFixedHamiltonianPathAlgorithm($G, v, v$)
\\\\
A Hamiltonian Cycle is a special case of a Hamiltonian Path where the start and end vertices happen to be the same vertex. This algorithm takes polynomial time because it makes at most $n$ calls to the (assumed polynomial) DoubleFixedHamiltonianPathAlgorithm.

\subsection*{SingleFixedHamiltonianPath $\leq_p$ DoubleFixedHamiltonianPath}
SingleFixedHamiltonianPath($G, u$):\\
\tab return $\bigvee_{v \in G}$ DoubleFixedHamiltonianPathAlgorithm($G, u, v$)
\\\\
A Single Fixed Hamiltonian Path can be discovered with a Double Fixed Hamiltonian Path algorithm by trying all possible endpoint vertices and seeing if a hamiltonian path exists between those two vertices. This algorithm takes polynomial time because it makes at most $n$ calls to the (assumed polynomial) DoubleFixedHamiltonianPathAlgorithm.

\subsection*{DoubleFixedHamiltonianPath $\leq_p$ HamiltonianCycle}
DoubleFixedHamiltonianPathAlgorithm($G, u, v$):\\
\tab return HamiltonianCycleAlgorithm($G + (u, v)$)
\\\\
All Hamiltonian Cycles are simple Hamiltonian Paths with an extra edge from the start vertex to the end vertex. Therefore, if there exists a Hamiltonian Cycle in the graph $G'$ which contains an extra edge from the start vertex to the end vertex, then there also contains a Hamiltonian Path in $G$ without the extra edge. This algorithm takes polynomial time because it makes at most $n$ calls to the (assumed polynomial) HamiltonianCycleAlgorithm.

\subsection*{DoubleFixedHamiltonianPath $\leq_p$ SingleFixedHamiltonianPath}
DoubleFixedHamiltonianPathAlgorithm($G, u, v$):\\
\tab Let $nexts = \{v\}$
\tab while $nexts \neq \phi$ \\
\tab\tab Let $next = nexts.pop()$ \\
\tab\tab if SingleFixedHamitonianPathAlgorithm($G - next, u$) $\land nexts = \phi$:\\
\tab\tab\tab return False\\
\tab\tab $G = G - next$\\
\tab\tab $nexts = nexts \union next.unvisitedNeighbors$ except $u$
\tab return True
\\\\
If there exists a Hamiltonian path in $G$ from $u$ to $v$, then there must exist a Hamiltonian path in $G - v$ from $u$ to one of $v$'s neighbors. Otherwise, there was never a Hamiltonian path in $G$ from $u$ to $v$ to begin with. This algorithm visits each of $v$'s neighbors, and neighbors of neighbors, and so on, and if all vertices are visited without a Hamiltonian path being deemed impossible, then it can return true. Otherwise, at some point a Hamiltonian path from $u$ to $v$ is impossible, so return false. This algorithm takes polynomial time because it makes at most $n$ calls to the (assumed polynomial) SingleFixedHamiltonianPathAlgorithm.

\subsection*{Summary}
If a Double Fixed Hamiltonian Path algorithm is poly-time, then it is possible to make both a poly-time Hamiltonian Cycle algorithm and a poly-time Single Fixed Hamiltonian Path algorithm.\\
If a Single Fixed Hamiltonian Path algorithm is poly-time, then it can make a poly-time Double Fixed Hamiltonian Path algorithm, which in turn can make a poly-time Hamiltonian Cycle algorithm. \\
If a Hamiltonian Cycle algorithm is poly-time, then it can make a poly-time Double Fixed Hamiltonian Path Algorithm, which in turn turn can make a poly-time Single Fixed Hamiltonian Path.

\section*{10}
In order to prove Hamiltonian Cycle is self reducible, we must show that we can use HamiltonianCycleDecisionAlgorithm to output a list of edges constituting a HC in $G$, in polynomial time.
\\\\
OptimalHamiltonianCycle($G$):
\tab if not HamiltonianCycleDecision($G$):\\
\tab\tab return False
\\\\
\tab for each edge $e \in G$:
\tab\tab if not HamiltonianCycleDecisionAlgorithm($G - e$):\\
\tab\tab\tab $path = path + e$\\
\tab\tab else:\\
\tab\tab\tab $G = G - e$\\
\tab Return $path$
The general strategy of this algorithm is to look at an edge $e$ in $G$, determine if we can still form an HC in $G$ when we remove $e$. If so, we can safely add $e$ to our solution. If not, we can exclude $e$. We repeat this until there are no edges left in $G$.
The number of times $HCD$ will be called inside of $HCO$ is at most $n$, where $n$ is the number of edges in $G$. Thus, we have proven that Hamiltonian Cycle is self reducible, since if we can determine whether a graph has an Hamiltonian Cycle in polynomial time, then we can find the Hamiltonian Cycle in polynomial time.  
\section*{12}
Vertex Cover is self-reducible if Optimal Vertex Cover $\leq_p$ Vertex Cover Decision. An algorithm for Optimal Vertex Cover takes as input a graph $G$ and returns $k$ vertices where $k$ is the smallest number of vertices needed for a vertex cover.
\\\\
OptimalVertexCoverAlgorithm($G$):\\
\# First, find the minimum number of vertices needed for a vertex cover by continually incrementing the number of vertices allowed until a vertex cover possible.
\tab Let $k = 0$\\
\tab while !VertexCoverDecision($G, k$):\\
\tab\tab $k = k + 1$
\\\\
\tab for each $v \in G$:\\
\tab\tab \# Try removing $v$ and all edges adjacent to $v$ in the graph. That is, assume $v$ is in a solution to the Vertex Cover.\\
\tab\tab if VertexCoverDecisionAlgorithm($G - v, k - 1$): \# If a vertex cover is possible with the rest of the graph,\\
\tab\tab\tab $S = S \cup \{v\}$ \# Then $v$ was a viable vertex to cover in the optimal solution, so append it to the solution set.\\
\tab\tab\tab \# Continue with the reduced problem size\\
\tab\tab\tab $G = G - v$\\
\tab\tab\tab $k = k - 1$
\\\\
\tab Return $S$
\\\\
The number of times VertexCoverDecisionAlgorithm will be called inside OptimalVertexCoverAlgorithm will be at most $2n$, where $2n$ is the number of vertices in $G$. Thus, if VertexCoverDecisionAlgorithm has a polynomial time algorithm, then so does OptimalVertexCoverAlgorithm, since $2n$ calls to a poly-time algorithm is still poly-time. So Vertex Cover is self reducible. 
\end{document}
