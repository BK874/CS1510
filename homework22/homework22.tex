\documentclass[letterpaper,notitlepage,twoside]{article}

% Basic imports, increase margins...
\usepackage[margin=0.75in]{geometry}
% Finite State Machine stuff

% Format tables nicely
\usepackage[latin1]{inputenc}
\usepackage{array}

\usepackage{amsfonts}
\usepackage{amssymb}
\usepackage{amsmath,amsthm}

\renewcommand{\implies}{\Rightarrow} % redefine command "implies"
\renewcommand{\iff}{\Leftrightarrow} % double arrow
\newcommand{\maps}{\rightarrow} % define command "map"
\newcommand{\union}{\cup}
\newcommand{\intersect}{\cap}
\newcommand{\N}{\mathbb{N}} % natural number
\newcommand{\Q}{\mathbb{Q}} % rational number
\newcommand{\R}{\mathbb{R}} % real number
\newcommand{\Z}{\mathbb{Z}} % integers
\newcommand\tab[1][1cm]{\hspace*{#1}} %\tab command

% Add more packages that you use here...

\begin{document}
\title{Homework 22}
\author{Brian Knotten, Brett Schreiber, Brian Falkenstein}
\maketitle
\section*{8}
\subsection*{HamiltonianCycle $\leq_p$ DoubleFixedHamiltonianPath}
HamiltonianCycleAlgorithm($G$):\\
\tab return $\bigvee_{v \in G)$ DoubleFixedHamiltonianPathAlgorithm($G, v, v$)
\\\\
A Hamiltonian Cycle is a special case of a Hamiltonian Path where the start and end vertices happen to be the same vertex. This algorithm takes polynomial time because it makes at most $n$ calls to the (assumed polynomial) DoubleFixedHamiltonianPathAlgorithm.

\subsection*{SingleFixedHamiltonianPath $\leq_p$ DoubleFixedHamiltonianPath}
SingleFixedHamiltonianPath($G, u$):\\
\tab return $\bigvee_{v \in G}$ DoubleFixedHamiltonianPathAlgorithm($G, u, v$)
\\\\
A Single Fixed Hamiltonian Path can be discovered with a Double Fixed Hamiltonian Path algorithm by trying all possible endpoint vertices and seeing if a hamiltonian path exists between those two vertices. This algorithm takes polynomial time because it makes at most $n$ calls to the (assumed polynomial) DoubleFixedHamiltonianPathAlgorithm.

\subsection*{DoubleFixedHamiltonianPath $\leq_p$ HamiltonianCycle}
DoubleFixedHamiltonianPathAlgorithm($G, u, v$):\\
\tab return $\bigvee_{(u', v') \notin G} HamiltonianCycleAlgorithm($G + (u', v')$)
\\\\
All Hamiltonian Cycles are simple Hamiltonian Paths with an extra edge from the start vertex to the end vertex. Therefore, if there exists a Hamiltonian Cycle in the graph $G'$ which contains an extra edge from the start vertex to the end vertex, then there also contains a Hamiltonian Path in $G$ without the extra edge. This algorithm takes polynomial time because it makes at most $n$ calls to the (assumed polynomial) HamiltonianCycleAlgorithm..

\section*{10}
Define $HCD(G)$ to be the algorithm for the decision problem for if a Hamiltonian Cycle exists for a graph $G$, and $HCO(G)$ to be optimization problem for actually finding the Hamiltonian Cycle in graph $G$. That is $HCD(G)$ will output 1 if an HC exists in $G$, and a 0 if not, and $HCO(G)$ will actually output the edges that constitute a HC in $G$, or 0 if one doesn't exist. The claim is that $HCO(G) \leq_p HCD(G)$, IE Hamiltonian Cycle is self reducible. \\
In order to prove this, we must show that we can use $HCD(G)$ to output a list of edges constituting a HC in $G$, in polynomial time. \\
First, assume graph $G$ is defined as a list of vertices and a list of edges. Consider the following pseudo-code: 
\begin{verbatim}
HCO(V, E):
      if HCD(V, E):                                    #initial check, make sure G has an HC
           HC = []                                     #initialize solution 
           while E.hasNext:                           #continue until no edges left 
                testEdge = E.pop                      #remove an edge from the graph
                if HCD(V, E):                         #check if HC exists in G minus one edge
                      HC.append(testEdge)            #if so, add the edge we removed to solution
           return isHC(HC, V, E)                       #function to determine if a path is a HC for a graph
     return 0

\end{verbatim}
The general strategy of this algorithm is to look at an edge $e$ in $G$, determine if we can still form an HC in $G$ when we remove $e$. If so, we can safely add $e$ to our solution. If not, we can exclude $e$. We repeat this until there are no edges left in $G$, and then we test if the cycle we've found is actually a HC. Note that $isHC$ could be defined very simply by checking that:
\begin{itemize}
\item All edges in $HC$ exist in $E$
\item No vertex in $V$ is visited more than once in $HC$
\item $HC$ spans all vertices in $V$
\end{itemize}
The number of times $HCD$ will be called inside of $HCO$ is at most $n$, where $n$ is the number of edges in $G$, as each time it is called at least 1 edge is removed. Similarly, $isHC$ will take at most $n$ time, as if $HC$ is a hamiltonian cycle, the max edges it could contain will be $n$. This results in a total run time of $n + n = O(n)$, a polynomial. Thus, we have proven that Hamiltonian Cycle is self reducible, and if we can determine whether a graph has an HC in polynomial time, we can find the HC in polynomial time.  
\section*{12}
Vertex Cover is self-reducible if Optimal Vertex Cover $\leq_p$ Vertex Cover Decision. An algorithm for Optimal Vertex Cover takes as input a graph $G$ and returns $k$ vertices where $k$ is the smallest number of vertices needed for a vertex cover.
\\\\
OptimalVertexCoverAlgorithm($G$):\\
# First, find the minimum number of vertices needed for a vertex cover by continually incrementing the number of vertices allowed until a vertex cover possible.
\tab Let $k = 0$\\
\tab while !VertexCoverDecision($G, k$):\\
\tab\tab $k = k + 1$
\\\\
\tab for each $v \in G$:\\
\tab\tab # Try removing $v$ and all edges adjacent to $v$ in the graph. That is, assume $v$ is in a solution to the Vertex Cover.\\
\tab\tab if VertexCoverDecisionAlgorithm($G - v, k - 1$): # If a vertex cover is possible with the rest of the graph,\\
\tab\tab\tab $S = S \cup \{v\}$ # Then $v$ was a viable vertex to cover in the optimal solution, so append it to the solution set.\\
\tab\tab\tab # Continue with the reduced problem size\\
\tab\tab\tab $G = G - v$\\
\tab\tab\tab $k = k - 1$
\\\\
\tab Return $S$
\\\\
The number of times VertexCoverDecisionAlgorithm will be called inside OptimalVertexCoverAlgorithm will be at most $2n$, where $2n$ is the number of vertices in $G$. Thus, if VertexCoverDecisionAlgorithm has a polynomial time algorithm, then so does OptimalVertexCoverAlgorithm, since $2n$ calls to a poly-time algorithm is still poly-time. So Vertex Cover is self reducible. 
\end{document}
