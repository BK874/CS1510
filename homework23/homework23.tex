\documentclass[letterpaper,notitlepage,twoside]{article}

% Basic imports, increase margins...
\usepackage[margin=0.75in]{geometry}
% Finite State Machine stuff

% Format tables nicely
\usepackage[latin1]{inputenc}
\usepackage{array}

\usepackage{amsfonts}
\usepackage{amssymb}
\usepackage{amsmath,amsthm}

\renewcommand{\implies}{\Rightarrow} % redefine command "implies"
\renewcommand{\iff}{\Leftrightarrow} % double arrow
\newcommand{\maps}{\rightarrow} % define command "map"
\newcommand{\union}{\cup}
\newcommand{\intersect}{\cap}
\newcommand{\N}{\mathbb{N}} % natural number
\newcommand{\Q}{\mathbb{Q}} % rational number
\newcommand{\R}{\mathbb{R}} % real number
\newcommand{\Z}{\mathbb{Z}} % integers
\newcommand\tab[1][1cm]{\hspace*{#1}} %\tab command

% Add more packages that you use here...

\begin{document}
\title{Homework 23}
\author{Brian Knotten, Brett Schreiber, Brian Falkenstein}
\maketitle
\subsection*{14}

\subsection*{16}
A 3-COLOR algorithm makes a choice at every vertex $v_i$ either to color it red, green, or blue. We can model that in a Boolean formula as three variables $r_i$, $g_i$, $b_i$, such that only one of the three can be true. So in the formula $F$ there is a clause $(r_i \oplus g_i \oplus b_i)$, which in CNF is:
\begin{align*}
(\neg r_i \lor \neg g_i) &\land \\
(\neg g_i \lor \neg b_i) &\land \\
(\neg b_i \lor \neg r_i) &\land \\
(r_i \lor g_i \lor b_i) 
\end{align*}
\subsection*{17}

\end{document}
