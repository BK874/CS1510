\documentclass[letterpaper,notitlepage,twoside]{article}

% Basic imports, increase margins...
\usepackage[margin=0.75in]{geometry}
% Finite State Machine stuff

% Format tables nicely
\usepackage[latin1]{inputenc}
\usepackage{array}

\usepackage{amsfonts}
\usepackage{amssymb}
\usepackage{amsmath,amsthm}

\renewcommand{\implies}{\Rightarrow} % redefine command "implies"
\renewcommand{\iff}{\Leftrightarrow} % double arrow
\newcommand{\maps}{\rightarrow} % define command "map"
\newcommand{\union}{\cup}
\newcommand{\intersect}{\cap}
\newcommand{\N}{\mathbb{N}} % natural number
\newcommand{\Q}{\mathbb{Q}} % rational number
\newcommand{\R}{\mathbb{R}} % real number
\newcommand{\Z}{\mathbb{Z}} % integers
\newcommand\tab[1][1cm]{\hspace*{#1}} %\tab command

% Add more packages that you use here...

\begin{document}
\title{Homework 23}
\author{Brian Knotten, Brett Schreiber, Brian Falkenstein}
\maketitle
\subsection*{14}

\subsection*{16}
A 3-COLOR algorithm makes a choice at every vertex $v_i$ either to color it red, green, or blue. We can model that in a Boolean formula as three variables $r_i$, $g_i$, $b_i$, such that only one of the three can be true. So in the formula $F$ there is a clause $(r_i \oplus g_i \oplus b_i)$, which in CNF is:
\begin{align*}
(\neg r_i \lor \neg g_i) &\land \\
(\neg g_i \lor \neg b_i) &\land \\
(\neg b_i \lor \neg r_i) &\land \\
(r_i \lor g_i \lor b_i) 
\end{align*}
Moreover, for each edge $(v_i, v_j) \in G$, it cannot be the case that they are the same color, which corresponds to the variables $r_i$ and $r_j$ not both being true, same with $g_i$ and $g_j$, and $b_i$ and $b_j$. So in the formula $F$ there is a clause $(\neg (r_i \land r_j)) \land \neg (g_i \land g_j) \land \neg (b_i \land b_j))$, which in CNF is:

\begin{align*}
(\neg r_i \lor \neg r_j) &\land \\
(\neg g_i \lor \neg g_j) &\land \\
(\neg b_i \lor \neg b_j)
\end{align*}

So an algorithm for 3-COLOR is:\\
3-COLORAlgorithm($G$):\\
\tab Let $F = \emptyset$ \\
\tab for each vertex $v_i \in G$: \\
\tab\tab $F = F \union$ (vertex CNF described above)
\\\\
\tab for each edge $(v_i, v_j) \in G$: \\
\tab\tab $F = F \union$ (edge CNF described above)
\\\\
\tab return CNF-SATAlgorithm($F$)
\\\\

It must be shown that $F$ is satisfiable if and only if $G$ is 3-Colorable.
\\\\
If $G$ is 3-Colorable, then each vertex is one of $\{r, g, b\}$. For each vertex $v_i$ colored $c$, let $c_i$ be true, and the other two colors associated with $i$ to be false in $F$, and this will satisfy $F$.
\\\\
If $F$ is satisfiable, then for all $i$, one of $r_i, g_i, b_i$ is true. Use this to color the vertex $v_i$ and it will give a valid 3-Coloring of $G$.
\\\\
Moreover, it must be shown that the transformation from $G$ to $F$ is polynomial. Since each vertex yields 4 clauses and each edge yields 3 clauses, which are both constant-size, the transformation is still polynomial in the number of vertices and number of edges, $O(n^2)$.
\subsection*{17}

\end{document}
