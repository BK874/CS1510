\documentclass[letterpaper,notitlepage,twoside]{article}

% Basic imports, increase margins...
\usepackage[margin=0.75in]{geometry}
% Finite State Machine stuff

% Format tables nicely
\usepackage[latin1]{inputenc}
\usepackage{array}

\usepackage{amsfonts}
\usepackage{amssymb}
\usepackage{amsmath,amsthm}

\renewcommand{\implies}{\Rightarrow} % redefine command "implies"
\renewcommand{\iff}{\Leftrightarrow} % double arrow
\newcommand{\maps}{\rightarrow} % define command "map"
\newcommand{\union}{\cup}
\newcommand{\intersect}{\cap}
\newcommand{\N}{\mathbb{N}} % natural number
\newcommand{\Q}{\mathbb{Q}} % rational number
\newcommand{\R}{\mathbb{R}} % real number
\newcommand{\Z}{\mathbb{Z}} % integers
\newcommand\tab[1][1cm]{\hspace*{#1}} %\tab command

% Add more packages that you use here...

\begin{document}
\title{Homework 23}
\author{Brian Knotten, Brett Schreiber, Brian Falkenstein}
\maketitle
\subsection*{14}
\subsection*{a}
This is an NP hard problem. The idea is to reduce Clique to an instance of Clique with a fixed $k$, specifically where $k=3n/4$. Call this new problem $3n/4Clique$. Obviously, if the $k$ for Clique already equals $3n/4$, no transformation needs to be made, and the output of $3n/4Clique$ would equal the output of $Clique$. \\
In the case where $k$ is greater than $3n/4$, we can construct $G'$ as input to  $3n/4Clique$ by adding vertices to $G$, such that all the added vertices have no edges connecting them to any other vertices, until $3n/4 = k$. These new vertices will obviously not contribute to any cliques existing in $G$, as they have no connections. 
Now, if $3n/4Clique$ returns true for $G'$, we know that there must be a clique of size $k$ in $G$, as $G'$ maintains all the cliques from $G$, with some added disconnected vertices to ensure that $k=3n/4$. IE it could not be the case that $G'$ has a clique of size $3n/4$ while $G$ does not have a clique of size $k$, as those $3n/4 -k$ vertices that were added could not contribute to any cliques. This transformation certainly takes polynomial time, as it simply requires adding a constant number of disconnected vertices.\\
In the case where $k$ is less than $3n/4$, we can construct $G'$ by adding cliques of size $(3n/4) - k$ to every vertex $V$, where each vertex in the new added clique connects to $V$, as well as all neighbor vertices to $V$ in $G$. Now, any vertex in $G$ that previously belonged to a clique of size $i$ belongs to a clique of size $i + (3n/4) - k$. Thus, if a clique of size $k$ exists in $G$, a clique of size $(3n/4) - k + k = 3n/4$ exists in $G'$. Further, if no clique of size $k$ exists in $G$, there won't be a clique of size $3n/4$ in $G'$, as none of the newly added cliques are connected to each other, and only add $(3n/4) - k$ to each already existing clique in $G$. This transformation will take polynomial time in the input, as we will need to add $n$ new cliques to the graph, each of constant size, and need to make at most $n$ new connections for each added clique (must connect to each vertices neighbors, vertices can have at most $n-1$ neighbors). \\
The pseudocode for this algorithm looks like:
\begin{verbatim}
Clique(G, k):
     if k < (3n/4):                                    //n = number of vertices in G
          G' = addCliques(G, (3n/4)-k)                 //poly time
          return FixedKClique(G')        
     else if k > (3n/4):
          G' = addDisconnectedVertices(G, k-(3n/4))  
          return FixedKClique(G')
     else:                                              //k = 3n/4
          return FixedKClique(G)         
\end{verbatim}
Thus, we have shown that if a poly time algorithm exists to determine if a clique of size $3n/4$ exists in a graph, then we can determine if a clique of size $k$ exists in a graph in poly time. 
\subsection*{b}
This is an NP hard problem. This problem is very similar to part $a$. That is, we will reduce IndependentSet to an instance of IndependentSet with $k=3n/4$. Again, if it is already the case that $k=3n/4$, no transformation need be made. \\
If $k>3n/4$, we can construct $G'$ by adding $k-(3n/4)$ new vertices to $G$, such that all the new vertices are connected to all other vertices. This includes other new vertices that were previously added. So, every new vertex $V'$ added to $G'$ will be connected to every other vertex in $G'$. Clearly, these new vertices cannot contribute to an independent set larger than size 1, as they are not independent of any other vertex in the graph. So if an independent set of size $k$ existed in the original $G$, it will still exist in $G'$, and our algorithm will return true (as $3n/4=k$ after the transformation). This transformation will be polynomial. We need a constant number of added vertices($k-(3n/4)$), and we must connect each added vertex to at most $n+k-(3n/4)$ other vertices. A constant times a polynomial plus a constant gives us polynomial run time.\\
If $k<3n/4$, we can construct $G'$ by adding $(3n/4) - k$ vertices that are disconnected from all other vertices in the graph. Now, any independent set if size $i$ that existed in $G$ has size $i +  (3n/4) - k$ in $G'$, as those added vertices can be added to any independent set without the risk of breaking the independency rule. This transformation is clearly polynomial, as we simply add a constant number of disconnected vertices. Thus, our algorithm will return true (G has an independent set of size $3n/4$) iff the original graph has an independent set of size $k$. \\
The pseudocode looks like:
\begin{verbatim}
IndependentSet(G, k):
      if k < (3n/4):
             G' = addDisconnectedVertices(G, (3n/4) - k)
             return FixedKIndependentSet(G')
       else if k > (3n/4):
              G' = addConnectedVertices(G, k-(3n/4))
              return FixedKIndependentSet(G')
       else:
              return FixedKIndependentSet(G)
\end{verbatim}
Thus, we have converted the input of IndependentSet to the input of the same problem with a fixed $k$ in polynomial time, and the output from the fixed $k$ problem can be converted back to the output of the general problem, so if a poly time algorithm exists for IndependentSet with the constraint that $k=3n/4$, so does a poly time algorithm exist for the general IndependentSet problem. 
\subsection*{c}
This is an NP-hard problem, because a poly-time algorithm for this problem implies a poly-time algorithm for the clique problem.
\\\\
CliqueAlgorithm($G, k$):\\
\tab Let $G' = G + k$ new vertices not connected to any other vertices.\\
\tab return CliqueAndIndependentSetAlgorithm($G, k$)
\\\\
The fact that $G'$ always has an independent set of $k$ vertices implies that CliqueAndIndependentSetAlgorithm on $G'$ will return true if and only if $G'$ (and therefore $G$) has a clique of size $k$. This algorithm is poly-time because the transformation adds only a linear number of vertices.

\subsection*{d}
This is an NP-hard problem. A poly time algorithm for CliqueOrIndependentSet implies a poly time algorithm for IndependentSet. Consider the following pseudocode:
\begin{verbatim}
IndependentSet(G, k):
     if CliqueOrIndependentSet(G, k):                            //initial check
           G' = addDisconnectedVertices(G, n)         //add n disconnected vertices
           return CliqueOrIndependentSet(G', k+n)
     return false                                           //if neither, return false
\end{verbatim}
Here, if CliqueOrIndependentSet returns true for the original $G$, we know that either a clique of size $k$ exists, or an independent set of size $k$ exists, or both. From here, we can transform $G$ so that the output of CliqueOrIndependentSet will tell us exactly which of clique or independent set of size $k$ existed in the original $G$. By adding $n$ disconnected vertices to $G$, we have ensured that any independent set of size $i$ in $G$ now has size $i + n$, as the $n$ disconnected vertices will be independent to all other vertices and can be added to any independent set to increase its size by $n$. However, since the max size a clique in $G$ could be is $n$, and the added vertices do not contribute to any cliques (other than the 1 clique they are a part of by themselves), the max clique size in $G'$ is still n, making a clique size of $k+n$ impossible. This ensures that if CliqueOrIndependentSet returns true, it cannot be the case that it is returning true for a clique of size $k+n$, and thus an independent set of size $k+n$ exists in $G'$, and an independent set of size $k$ existed in $G$. 
\subsection*{e}
This is an NP-hard problem, because a poly-time algorithm for this problem implies a poly-time algorithm for the 3n/4 clique problem.
\\\\
3n/4CliqueAlgorithm($G$):\\
\tab Let $G' = G + 3n/4$ new vertices not connected to any other vertices.\\
\tab return 3n/4CliqueAndIndependentSetAlgorithm($G$)
\\\\
The fact that $G'$ always has an independent set of $3n/4$ vertices implies that CliqueAndIndependentSetAlgorithm on $G'$ will return true if and only if $G'$ (and therefore $G$) has a clique of size $3n/4$. This algorithm is poly-time because the transformation adds only a linear number of vertices.
\subsection*{f}

\subsection*{16}
A 3-COLOR algorithm makes a choice at every vertex $v_i$ either to color it red, green, or blue. We can model that in a Boolean formula as three variables $r_i$, $g_i$, $b_i$, such that only one of the three can be true. So in the formula $F$ there is a clause $(r_i \oplus g_i \oplus b_i)$, which in CNF is:
\begin{align*}
(\neg r_i \lor \neg g_i) &\land \\
(\neg g_i \lor \neg b_i) &\land \\
(\neg b_i \lor \neg r_i) &\land \\
(r_i \lor g_i \lor b_i) 
\end{align*}
Moreover, for each edge $(v_i, v_j) \in G$, it cannot be the case that they are the same color, which corresponds to the variables $r_i$ and $r_j$ not both being true, same with $g_i$ and $g_j$, and $b_i$ and $b_j$. So in the formula $F$ there is a clause $(\neg (r_i \land r_j)) \land \neg (g_i \land g_j) \land \neg (b_i \land b_j))$, which in CNF is:

\begin{align*}
(\neg r_i \lor \neg r_j) &\land \\
(\neg g_i \lor \neg g_j) &\land \\
(\neg b_i \lor \neg b_j)
\end{align*}

So an algorithm for 3-COLOR is:\\
3-COLORAlgorithm($G$):\\
\tab Let $F = \emptyset$ \\
\tab for each vertex $v_i \in G$: \\
\tab\tab $F = F \union$ (vertex CNF described above)
\\\\
\tab for each edge $(v_i, v_j) \in G$: \\
\tab\tab $F = F \union$ (edge CNF described above)
\\\\
\tab return CNF-SATAlgorithm($F$)
\\\\

It must be shown that $F$ is satisfiable if and only if $G$ is 3-Colorable.
\\\\
If $G$ is 3-Colorable, then each vertex is one of $\{r, g, b\}$. For each vertex $v_i$ colored $c$, let $c_i$ be true, and the other two colors associated with $i$ to be false in $F$, and this will satisfy $F$.
\\\\
If $F$ is satisfiable, then for all $i$, one of $r_i, g_i, b_i$ is true. Use this to color the vertex $v_i$ and it will give a valid 3-Coloring of $G$.
\\\\
Moreover, it must be shown that the transformation from $G$ to $F$ is polynomial. Since each vertex yields 4 clauses and each edge yields 3 clauses, which are both constant-size, the transformation is still polynomial in the number of vertices and number of edges, $O(n^2)$.
\subsection*{17}
In order to reduce the Vertex Cover problem to the Dominating Set problem, given the graph $G$ and desired cover size $k$, we will construct a graph $G'$ such that $G'$ has a dominating set of size $k$ iff $G$ has a vertex cover of size $k$. \\
Let $V$ and $E$ be the set of all vertices and edges in $G$ respectively. We will construct $G'$ as follows:
\\\\
$V'$ contains all the vertices in $V$ as well as new utility vertices corresponding to each edge in $E$. \\
$V' = \{v | v \in V\} \union \{ v' | (u, v) \in E\}$
\\\\
$E'$ contains all the edges in $V$ as well as new edges connecting each utility vertex to the endpoints of the edge it was derived from. Moreover, all vertices originally in $V$ have new edges between them.\\
$E' = \{(u, v), (u, v'), (v', v) | (u, v) \in E\} \union \{(u, v) | (u, v) \in V \times V \land u \neq v \}$.\\\\
To prove that $G'$ has a dominating set of size $k$ iff $G$ has a vertex cover of size $k$ we must prove the relationship in both directions: \\
First, assume $G$ has a vertex cover $VC$ of size $k$. Then $G'$ has a dominating set $DS = VC$ of size $k$ as every vertex $v \in VC$ has a representative vertex $u$ in $G'$ that is connected to every other vertex in $G'$ representing a vertex in $G$ and every vertex in $G'$ representing an edge in $G$ must connected to at least one vertex in $VC$ because at least one of that edge's endpoints in $G$ must be a vertex in $VC$ by the definition of a vertex cover. \\\\
Second, assume $G'$ has a dominating set $DS$ of size $k$. If $DS$ contains any vertices representing an edge in $G$, replace it with one of the vertices representing an endpoint of its edge in $G$. $DS$ is still a dominating set because any edge-representing vertex removed is next to its replacement (one of its endpoints) and every vertex that was next to the removed vertex is also next to its replacement (every endpoint is connected to every other endpoint by our construction). By definition, the vertices of $G'$ are all either in $DS$ or are connected by an edge to a vertex in $DS$. We can therefore construct a vertex cover $VC$ of $G$ by taking every vertex in $G$ that is a vertex represented by a vertex in $DS$ i.e. $VC = DS$. $VC$ must be a vertex cover of $G$ as for any edge in $G$ there must be a vertex in $DS$ that is connected to the vertex representing that edge in $G'$ by the definition of a dominating set. Because $VC = DS$, that vertex is one of the endpoints of that edge in $G$ and is in $VC$.

\end{document}
