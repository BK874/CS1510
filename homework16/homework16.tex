\documentclass[letterpaper,notitlepage,twoside]{article}

% Basic imports, increase margins...
\usepackage[margin=0.75in]{geometry}
% Finite State Machine stuff

% Format tables nicely
\usepackage[latin1]{inputenc}
\usepackage{array}

\usepackage{amsfonts} 
\usepackage{amssymb}
\usepackage{amsmath,amsthm}

\renewcommand{\implies}{\Rightarrow} % redefine command "implies"  
\renewcommand{\iff}{\Leftrightarrow} % double arrow
\newcommand{\maps}{\rightarrow} % define command "map" 
\newcommand{\union}{\cup}
\newcommand{\intersect}{\cap}
\newcommand{\N}{\mathbb{N}} % natural number 
\newcommand{\Q}{\mathbb{Q}} % rational number 
\newcommand{\R}{\mathbb{R}} % real number 
\newcommand{\Z}{\mathbb{Z}} % integers 
\newcommand\tab[1][1cm]{\hspace*{#1}} %\tab command

% Add more packages that you use here...

\begin{document}
\title{Homework 15}
\author{Brian Knotten, Brett Schreiber, Brian Falkenstein}
\maketitle

\section*{20}
A decision tree is as follows:
The decision tree can be thought of as follows for each node $v$ at depth $i$:
\begin{itemize}
  \item $v$ designates each of $p_1,p_2,...p_i$ into lists $A$ and $B$, which represent the routes that either taxi $A$ or taxi $B$ travels.
  \item $v$'s left child copies the assignments of $v$ to $A$ and $B$, but appends point $p_{i + 1}$ to $A$.
  \item $v$'s right child copies the assignments of $v$ to $A$ and $B$, but appends point $p_{i + 1}$ to $B$.
  \item At depth $i$, $i$ points have been visited.
\end{itemize}
Because each point needs to be visited by at least one taxi, and since there are no benefits for having both taxis visit a point, each point is either visited by taxi $A$ or by taxi $B$. So there are $2^n$ possible point designations at the leaves of the decision tree. \\
The decision tree can be pruned using the following rules:
\begin{enumerate}
  \item If two nodes $u$ and $v$ have the same total distance, $dist(u_A) + dist(u_B) = dist(v_A) + dist(v_B)$, prune $v$.
  
\end{enumerate}

\begin{verbatim}
taxi(i, a, b):
    if i > n:
        return 0
    
    return min(
        dist(p[a], p[i]) + taxi(i + 1, i, b)
        dist(p[b], p[i]) + taxi(i + 1, a, i)
    )
\end{verbatim}

Let $taxi[i, a, b] =$ the minimum total distance of designating $A$ and $B$ over points $p_1...p_i$, and $A$'s last stop is $p_a$, and $B$'s last stop is $p_b$. The dynamic solution algorithm is:

\begin{verbatim}
for a = 1 to n do:
    for b = 1 to n do:
        for i = n to 1 do:
        
            taxi[a, b, i] = min(
                                taxi[i, b, i + 1] + dist(p[a], p[i]),
					                      taxi[a, i, i + 1] + dist(p[b], p[i])
				                    )
\end{verbatim}
\section*{21}
The decision tree can be defined as: 
\begin{itemize}
\item Every node $v$ contains the cumulative response time up until that point in the path, and the path taken to get to $v$. 
\item Every node having at most $n$ children, where each child represents the next point to travel to
\item The leaves being paths covering all $n$ points
\item The solution being the minimum response time found at the leaves
\end{itemize}
Without pruning, the size of the tree will be $n^n$ (height of $n$ since every path must cover all points, and a branching factor of $n$). The tree can then be pruned using the following rules:
\begin{enumerate}
\item If a node visits a point in its path that it has already visited, prune it.
\item If 2 nodes at the same depth have visited the same points in its path, prune the one with the longer cumulative response time.
\end{enumerate}
\end{document}
