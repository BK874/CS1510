\documentclass[letterpaper,notitlepage,twoside]{article}

% Basic imports, increase margins...
\usepackage[margin=0.75in]{geometry}
% Finite State Machine stuff

% Format tables nicely
\usepackage[latin1]{inputenc}
\usepackage{array}

\usepackage{amsfonts} 
\usepackage{amssymb}
\usepackage{amsmath,amsthm}

\renewcommand{\implies}{\Rightarrow} % redefine command "implies"  
\renewcommand{\iff}{\Leftrightarrow} % double arrow
\newcommand{\maps}{\rightarrow} % define command "map" 
\newcommand{\union}{\cup}
\newcommand{\intersect}{\cap}
\newcommand{\N}{\mathbb{N}} % natural number 
\newcommand{\Q}{\mathbb{Q}} % rational number 
\newcommand{\R}{\mathbb{R}} % real number 
\newcommand{\Z}{\mathbb{Z}} % integers 
\newcommand\tab[1][1cm]{\hspace*{#1}} %\tab command

% Add more packages that you use here...

\begin{document}
\title{Homework 15}
\author{Brian Knotten, Brett Schreiber, Brian Falkenstein}
\maketitle

\section*{20}
The decision tree can be thought of as:
\begin{itemize}
  \item Each node has a branching factor of 2 and stores the cumulative distance traveled by the taxis (separately) up to this point.
  \item The left child of a node a level $i$ represents taxi $A$ visiting the point $p_{i+1}$ and the right child represents taxi $B$ visiting the point $p_{i+1}$. (why would both visit? edge cases? - at best it would add no extra distance?)
  \item At depth $i$, $i$ points have been visited.
  \item The root of the tree would represent no nodes having been visited by either taxi and would therefore have 0 for the distance traveled by both taxis.
\end{itemize}
Because each point needs to be visited by at least one taxi, and there are no discernible benefits(?) for having both taxis visit a point, each point is either visited by taxi $A$ or by taxi $B$. Therefore there are $2^n$ possible paths and the decision tree described above enumerates all $2^n$ possible path pairs through the points in order. \\
The decision tree can be pruned an transformed into a dynamic pruning algorithm using the following rules:
\begin{enumerate}
  \item
  
\end{enumerate}
  
\section*{21}

\end{document}
