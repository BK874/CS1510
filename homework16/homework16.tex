\documentclass[letterpaper,notitlepage,twoside]{article}

% Basic imports, increase margins...
\usepackage[margin=0.75in]{geometry}
% Finite State Machine stuff

% Format tables nicely
\usepackage[latin1]{inputenc}
\usepackage{array}

\usepackage{amsfonts} 
\usepackage{amssymb}
\usepackage{amsmath,amsthm}

\renewcommand{\implies}{\Rightarrow} % redefine command "implies"  
\renewcommand{\iff}{\Leftrightarrow} % double arrow
\newcommand{\maps}{\rightarrow} % define command "map" 
\newcommand{\union}{\cup}
\newcommand{\intersect}{\cap}
\newcommand{\N}{\mathbb{N}} % natural number 
\newcommand{\Q}{\mathbb{Q}} % rational number 
\newcommand{\R}{\mathbb{R}} % real number 
\newcommand{\Z}{\mathbb{Z}} % integers 
\newcommand\tab[1][1cm]{\hspace*{#1}} %\tab command

% Add more packages that you use here...

\begin{document}
\title{Homework 15}
\author{Brian Knotten, Brett Schreiber, Brian Falkenstein}
\maketitle

\section*{20}
A decision tree is as follows:
The decision tree can be thought of as follows for each node $v$ at depth $i$:
\begin{itemize}
  \item $v$ designates each of $p_1,p_2,...p_i$ into lists $A$ and $B$, which represent the routes that either taxi $A$ or taxi $B$ travels.
  \item $v$'s left child copies the assignments of $v$ to $A$ and $B$, but appends point $p_{i + 1}$ to $A$.
  \item $v$'s right child copies the assignments of $v$ to $A$ and $B$, but appends point $p_{i + 1}$ to $B$.
  \item At depth $i$, $i$ points have been visited.
\end{itemize}
Because each point needs to be visited by at least one taxi, and since there are no benefits for having both taxis visit a point, each point is either visited by taxi $A$ or by taxi $B$. So there are $2^n$ possible point designations at the leaves of the decision tree. \\
The decision tree can be pruned using the following rules:
\begin{enumerate}
  \item If two nodes $u$ and $v$ have the same total distance, $dist(u_A) + dist(u_B) = dist(v_A) + dist(v_B)$, prune $v$.
  
\end{enumerate}

\begin{verbatim}
taxi(i, a, b):
    if i > n:
        return 0
    
    return min(
        dist(p[a], p[i]) + taxi(i + 1, i, b)
        dist(p[b], p[i]) + taxi(i + 1, a, i)
    )
\end{verbatim}

Let $taxi[i, a, b] =$ the minimum total distance of designating $A$ and $B$ over points $p_1...p_i$, and $A$'s last stop is $p_a$, and $B$'s last stop is $p_b$. The dynamic solution algorithm is:

\begin{verbatim}
for a = 1 to n do:
    for b = 1 to n do:
        for i = n to 1 do:
        
            taxi[a, b, i] = min(
                                taxi[i, b, i + 1] + dist(p[a], p[i]),
					            taxi[a, i, i + 1] + dist(p[b], p[i])
				            )
\end{verbatim}
\section*{21}
The decision tree of depth $n$ can be defined as follows. Given a node $v$ at depth $i$: 
\begin{itemize}
\item $v$ contains a list of stops of length $i$ and the cumulative response time to make $i$ stops. 
\item $v$ has $n$ children, where the $j$th child cointains $v$'s path appended with $x_j$.
\end{itemize}
Therefore, the solution is the leaf node (a path of $n$ stops) with the minimum average response time. Without pruning, the size of the tree will be $n^n$ (height of $n$ since every path must cover all points, and a branching factor of $n$). The tree can then be pruned using the following rules:
\begin{enumerate}
\item If a node $v$ visits a point in its path that it has already visited, prune $v$.
\item If nodes $u$ and $v$ are at the same depth and have visited the same points, and $u$ has an equal or shorter average response time, then prune $v$.
\end{enumerate}

The first pruning rule ensures that every node $v$ at depth $i$ only has $n - i$ children, since $i$ children are already somewhere in $v$'s path.
\\\\
The second pruning rule ensures that at every depth $i$, there will only be one node $v$ whose path goes through the set of $i$ points. That is, every node at a given depth are only distinguished by the points they go through.

$A[i, j] =$ The average response time making $i$ stops in total, and ending at point $x_j$.

A supplementary array Visited is needed, where $Visited[i] = 1$ if point $x_i$ has been visited during the algorithm, otherwise $Visited[i] = 0$.

\begin{verbatim}
for i = 1 to n do:
    Visited[i] = 0

for i = 1 to n do:           # For every next stop $i$
    
    for j = 1 to n do:       # For every point being considered:
        if Visited[j] = 1:   # If that point has already been added,
            continue         # Skip it


        if A[i - 1, j] < A[i - 1, minimum]:
            minimum = j


    A[i, minimum] = 0
    Visited[minimum] = 1


Output A[n, k] where k produces the minimum average waiting time. 

\end{verbatim}

\end{document}
