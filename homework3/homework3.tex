\documentclass[letterpaper,notitlepage,twoside]{article}

% Basic imports, increase margins...
\usepackage[margin=0.75in]{geometry}
\usepackage{amssymb}
\usepackage{amsmath}

% Format tables nicely
\usepackage[latin1]{inputenc}
\usepackage{array}
\usepackage{booktabs}
\setlength{\heavyrulewidth}{1.5pt}
\setlength{\abovetopsep}{4pt}

\usepackage{amsfonts} 
\usepackage{amssymb}
\usepackage{amsmath,amsthm}

\renewcommand{\implies}{\Rightarrow} % redefine command "implies"  
\renewcommand{\iff}{\Leftrightarrow} % double arrow
\newcommand{\maps}{\rightarrow} % define command "map" 
\newcommand{\union}{\cup}
\newcommand{\intersect}{\cap}
\newcommand{\N}{\mathbb{N}} % natural number 
\newcommand{\Q}{\mathbb{Q}} % rational number 
\newcommand{\R}{\mathbb{R}} % real number 
\newcommand{\Z}{\mathbb{Z}} % integers 
\newcommand\tab[1][1cm]{\hspace*{#1}} %\tab command

% Add more packages that you use here..

\title{CS 1510 Homework 2}
\author{Brian Falkenstein, Brian Knotten, Brett Schreiber}

\begin{document}
\maketitle

\section*{4}
\subsection*{a}
\begin{itemize}
	\item Let $A$ be an algorithm which produces a list fill amounts a motorist takes at each stop. 
	\item For sake of contradiction assume $A$ is not correct.
	\item Therefore there exists an input $I$ where $A$ produces a non-optimal output.
	\item Let $opt(I)$ be the optimal output that differs the least from $A(I)$.
	\item Since $A(I) \neq opt(I)$, there exists a first item in the list at stop $x_i$ where $A(I)$'s fill amount $f$ differs from $opt(I)$'s fill amount $g$.
	\item Every prior fill amount before $x_i$ must be identical between $A(I)$ and $opt(I)$.
	\item Since $f \neq g$, then $f > g$ or $f < g$.
	\item If $f > g$:
	\begin{itemize}
		\item Since $opt(I)$ is correct, $g$ was enough fuel to get to the next stop.
		\item Therefore $f$ was not the minimum amount of gas needed.
		\item $A$ guarantees only filling up the minimum amount of gas needed.
		\item There is a contradiction.
	\end{itemize}
	
	\item So it can only be the case that $f < g$.
	\item $A(I)$ spends less time ( $\frac{g - f}{r}$ seconds) filling up at $x_i$.
	\item Since $A(I)$ is correct (yet suboptimal), $f$ was enough fuel to get to the next stop.
	\item Consider an alternative optimal solution $opt'(I)$ identical to $opt(I)$ except fill amount $g$ is replaced with $f$ at $x_i$.
	\item The time is still optimal, since less time is spent fueling at $x_i$. ($\frac{g - f}{r}$) seconds less.
	\item $opt'(I)$ cannot use less time than $opt(I)$, otherwise $opt(I)$ would not be optimal. So $opt'(I)$ utilizes $\frac{g - f}{r}$ seconds somewhere else.
	\item This extra time can be used to ensure that $opt'(I)$ is still a correct algorithm.
\end{itemize}


\subsection*{b}
This algorithm $A$ does not provide an optimal output for the following input $I$:\\
\begin{itemize}
	\item Let $A$ be the first gas station at kilometer 0
	\item Let $x$ be the second gas station at kilometer 2
	\item Let $B$ be the destination at kilometer 4
	\item Let $C$ be the capacity 3 liters
	\item Let $F$ be the consumption rate of 1 liter per kilometer
	\item Let $r$ be the fill rate of 1 liter per minute
\end{itemize}
$A(I)$ produces the following output (a list of actions):
\begin{enumerate}
	\item Fuel tank starts at 0/3.
	\item Not enough gas to make it to $x$, so fill the tank up all the way with 3 liters at $A$, taking 3 minutes. (Fuel tank at 3/3).
	\item Travel to gas station $x$, 2 kilometers away. (Fuel tank at 1/3).
	\item Not enough gas to make it to $B$, so fill the tank up all the way with 2 liters at $x$ taking 2 minutes. (Fuel tank at 3/3).
	\item Arrive at destination $B$, 2 kilometers away. (Fuel tank at 1/3).
\end{enumerate}
$A(I)$ requires 3 + 2 = 5 minutes of fueling.

A better output $O(I)$ is:
\begin{enumerate}
	\item Fuel tank starts at 0/3.
	\item Fill 3 liters at $A$, taking 3 minutes. (Fuel tank at 3/3).
	\item Travel to gas station $x$, 2 kilometers away. (Fuel tank at 1/3).
	\item Fill 1 liters at $x$, taking 1 minute. (Fuel tank at 1/3).
	\item Arrive at destination $B$, 2 kilometers away. (Fuel tank at 0/3).
\end{enumerate}
$O(I)$ requires 3 + 1 = 4 minutes of fueling.\\

Since there exists a more optimal output than $A(I)$, $A$ is not a correct algorithm.
\end{document}
