\documentclass[letterpaper,notitlepage,twoside]{article}

% Basic imports, increase margins...
\usepackage[margin=0.75in]{geometry}
% Finite State Machine stuff

% Format tables nicely
\usepackage[latin1]{inputenc}
\usepackage{array}

\usepackage{amsfonts} 
\usepackage{amssymb}
\usepackage{amsmath,amsthm}

\renewcommand{\implies}{\Rightarrow} % redefine command "implies"  
\renewcommand{\iff}{\Leftrightarrow} % double arrow
\newcommand{\maps}{\rightarrow} % define command "map" 
\newcommand{\union}{\cup}
\newcommand{\intersect}{\cap}
\newcommand{\N}{\mathbb{N}} % natural number 
\newcommand{\Q}{\mathbb{Q}} % rational number 
\newcommand{\R}{\mathbb{R}} % real number 
\newcommand{\Z}{\mathbb{Z}} % integers 
\newcommand\tab[1][1cm]{\hspace*{#1}} %\tab command

% Add more packages that you use here...

\begin{document}
\title{Homework 10}
\author{Brian Knotten, Brett Schreiber, Brian Falkenstein}
\maketitle


\subsection*{8}
Assume the vertices are ordered via a breadth first search or a depth first search. This ensures a list of vertices from $v_1$ to $v_n$ such that any vertex $v_i$ is connected to one and only one vertex in $v_1...v_{i-1}$. $v_i$ cannot be connected to more than one vertex in $v_1...v_{i-1}$, otherwise there would be a cycle and $T$ would not be a tree. $v_1$ must be connect to a vertex in $v_1...v_{i-1}$ by definition of a tree search. Let $neighbor(v_i)$ represent the sole neighbor of $v_i$ of $v_1...v_i$. Let $w_i$ be the weight \\
\\
\# Base case: When $T$ has one vertex, the most profitable (and only) thing to do is stay put. Return 0 for the best profit.\\
MostProfitableEndpoints[1] $= (1, 1)$\\
MaximumProfit[1] = 0\\
\\
MostProfitablePathRootedAt[1] $= v_1$ \# The path is (1, 1): the start and end point\\
RootedProfit[1] = 0\\
\\
for i := 2 to n do: \\
\tab if RootedProfit[neighbor($v_i$)]  $+ w_i > 0$ then: \\
\tab\tab RootedProfit[i] $=$ RootedProfit[neighbor($v_i)$] \\
\tab\tab MostProfitablePathRootedAt[i] =  MostProfitablePathRootedAt[neighbor($v_i$)] \\
\tab else: \# Stay put \\
\tab\tab RootedProfit[i] $= 0$ \\
\tab\tab MostProfitablePathRootedAt[i] $= v_i$ \\
\\
Let u, v = MostProfitableEndpoints[i - 1]\\
\tab if neighbor($v_i$) $\in$ (u, v) and $w_i > 0$ then: \# $v_i$ adds value to the optimal solution\\
\tab\tab if neighbor($v_i$) == u then: MostProfitableEndpoints[i] = ($v_i$, v)\\
\tab\tab else then neighbor($v_i$) == v: MostProfitableEndpoints[i] = (u, $v_i$)\\
\tab MaximumProfit[i] = MaximumProfit[i - 1] + $w_i$\\

\tab else if RootedProfit[i] > MaximumProfit[i - 1]: \# $v_i$ could possibly still be in the best path.\\
\tab\tab MaximumProfit[i] = RootedProfit[i]\\
\tab\tab MostProfitablePathRootedAt[i] = MostProfitablePathRootedAt[i - 1]\\
\\
\tab else: \# $v_i$ doesn't contribute at all to the maximum profit path\\
\tab\tab MaximumProfit[i] = MaximumProfit[i - 1]\\
\tab\tab MostProfitableEndpoints[i] = MostProfitableEndpoints[i - 1]\\
\\
Output MostProfitableEndpoints[n]

\section*{9}
An algorithm for this problem is a modified form of the Optimal Binary Search Tree dynamic algorithm. Like the OBST dynamic programming algorithm, this algorithm caches the minimum weight of a tree with keys $K_l$ and $K_r$ in a 2D array, Weight[l][r]. It also stores the key $K_i$ where $l \leq i \leq r$ that should be used as the root of the OBST for keys from $K_l$ to $K_r$ in a 2D array, Root[l][r]. Our algorithm differs from the OBST algorithm in that there is a third 2D array, Height[l][r] which stores the height of the OBST with keys from $K_l$ to $K_r$.\\
\\
The base case is the same as with the OBST algorithm. When $l = r$, that is, on the diagonal of the 2D array, the only key that can be considered as the (childless) root is $K_l$. So Root[l][r] = $K_l$ and Weight[l][r] = $w_l$. Moreover, since the sole key does not have any children, Height[l][r] = 1.
\\
The recursive case considers all keys from $K_l$ to $K_r$. Like the OBST algorithm, consider the key $K_i$ that has the minimum weight of the sum of its subtrees Weight[l][i-1] + Weight[i + 1][r] plus the sum of the weight of all the keys $K_l$ through $K_r$, and make Root[l][r] = $K_i$. But do not consider $K_i$ if Height[l][i - 1] - Height[i + 1][r] > 1 or < -1. Discard $K_i$ as a candidate root if this is the case.
\\
As with OBST, output the tree from $K_0$ to $K_n$. This algorithm is very similar, only it requires caching a third property of the subtrees, and has greater scrutiny over candidate roots.

\end{document}
