\documentclass[letterpaper,notitlepage,twoside]{article}

% Basic imports, increase margins...
\usepackage[margin=0.75in]{geometry}
\usepackage{amssymb}
\usepackage{amsmath}

% Finite State Machine stuff
\usepackage{pgf}
\usepackage{tikz}
\usetikzlibrary{arrows,automata}
\usepackage{pgfplots}

% Format tables nicely
\usepackage[latin1]{inputenc}
\usepackage{array}
\usepackage{booktabs}
\setlength{\heavyrulewidth}{1.5pt}
\setlength{\abovetopsep}{4pt}

\usepackage{amsfonts} 
\usepackage{amssymb}
\usepackage{amsmath,amsthm}

\renewcommand{\implies}{\Rightarrow} % redefine command "implies"  
\renewcommand{\iff}{\Leftrightarrow} % double arrow
\newcommand{\maps}{\rightarrow} % define command "map" 
\newcommand{\union}{\cup}
\newcommand{\intersect}{\cap}
\newcommand{\N}{\mathbb{N}} % natural number 
\newcommand{\Q}{\mathbb{Q}} % rational number 
\newcommand{\R}{\mathbb{R}} % real number 
\newcommand{\Z}{\mathbb{Z}} % integers 
\newcommand\tab[1][1cm]{\hspace*{#1}} %\tab command

% Add more packages that you use here..

\title{CS 1510 Homework 4}
\author{Brian Falkenstein, Brian Knotten, Brett Schreiber}

\begin{document}
\maketitle

\section*{9}
\subsection*{a}
This algorithm can be proven incorrect by considering a situation where large files, which by the algorithm will be placed after shorter files, have a higher probability of being accessed. This is demonstrated most simply in an example with 2 files, $F_1$ of size 2 and probability 0.1, and $F_2$ of size 10 and probability 0.9. Following the algorithm, $F_2$ would be placed after $F_1$, giving an expected access time of $(2*0.2)+(12*0.9)=11$. This can be improved by simply swapping the locations of the two files, giving us an expected access time of $(10*0.9)+(12*.1)=10.2$.
\subsection*{b}



\end{document}
