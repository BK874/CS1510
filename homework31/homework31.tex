
\documentclass[letterpaper,notitlepage,twoside]{article}

% Basic imports, increase margins...
\usepackage[margin=0.75in]{geometry}
% Finite State Machine stuff
\usepackage{pgf}
\usepackage{tikz}
\usetikzlibrary{arrows, automata}
% Format tables nicely
\usepackage[latin1]{inputenc}
\usepackage{array}

\usepackage{amsfonts}
\usepackage{amssymb}
\usepackage{amsmath,amsthm}

\usepackage{algorithm}          %  float wrapper for algorithms.
\usepackage{algpseudocode}      % layout for algorithmicx
\usepackage{float} % http:/ctan.org/pkg/flot - [H] float parameter forces a 'float' (algorithm block) to remain in a location

\renewcommand{\implies}{\Rightarrow} % redefine command "implies"
\renewcommand{\iff}{\Leftrightarrow} % double arrow
\newcommand{\maps}{\rightarrow} % define command "map"
\newcommand{\union}{\cup}
\newcommand{\intersect}{\cap}
\newcommand{\N}{\mathbb{N}} % natural number
\newcommand{\Q}{\mathbb{Q}} % rational number
\newcommand{\R}{\mathbb{R}} % real number
\newcommand{\Z}{\mathbb{Z}} % integers
\newcommand\tab[1][1cm]{\hspace*{#1}} %\tab command

% Add more packages that you use here...

\begin{document}
\title{Homework 31}
\author{Brian Knotten, Brett Schreiber, Brian Falkenstein}
\maketitle

\section*{2}
To give evidence that finding a fast efficient parallel algorithm (i.e. a $O(log^k(n))$ algorithm for some $k$ with a poly number of processors) for $N$ is at least as hard as finding a fast efficient parallel algorithm for the $BFV$ problem, you could present a parallel algorithm for the $BFV$ problem that is $O(log^k(n))$ for some $k$ with a poly number of processors that is only missing code for $N$. \\
The algorithm should convert the input from the $BFV$ problem to the input to the $N$ problem in at most poly-log time, run the missing code for $N$ using the converted input, and convert the output of the $N$ problem to the output of the $BFV$ in at most poly-log time. \\
Further, the missing code for $N$ should give a correct output given the input iff there is a correct output for the BFV problem given the input. \\
Given this parallel algorithm for the $BFV$missing the code for $N$, if you were to find parallel code for $N$ that runs in poly-log time with a poly number of processors then you would immediately have a poly-log algorithm for the $BFV$ problem with a poly number of processors by "plugging in" the code for $N$ into the parallel code for $BFV$. \\
Because the code we are plugging in is poly-log with a poly number of processors, our input/output transformations are also poly-log with a poly number of processors, and the code gives a correct output for $BFV$ iff the code for $N$ gives a correct output, the newly-constructed algorithm correctly solves the $BFV$ problem in poly-log time using a poly number of processors. \\
Therefore finding a fast efficient parallel algorithm for problem $N$ is at least as hard of a problem as finding a fast efficient parallel algorithm for the Boolean Formula Value problem. \\

\section*{16}

Given two lists of increasing items $a_0...a_{n - 1}$ and $b_1...b_{n - 1}$, the following must be true for the merged increasing list $M$ of size $2n$:
\begin{enumerate}
\item The first element of $M$ must be the minimum of $a_0$ and $b_0$
\item If the first element of $M$ is $a_0$, then the second element must be the minimum of $a_1$ and $b_0$. \\
      If the first element of $M$ is $b_0$, then the second element must be the minimum of $a_0$ and $b_1$.
\item If the second element of $M$ is $a_0$, then the first element must have been $b_0$, and so the third element must be the minimum of $a_1$ and $b_1$. \\
      If the second element of $M$ is $b_0$, then the first element must have been $a_0$, and so the third element must be the minimum of $a_1$ and $b_1$ as above. \\
	  If the second element of $M$ is $a_1$, then the first element must have been $a_0$, and so the third element must be the minimum of $a_2$ and $b_0$. \\
	  If the second element of $M$ is $b_1$, then the first element must have been $b_0$, and so the third element must be the minimum of $a_0$ and $b_2$.
\end{enumerate}

The third row could be any of $a_0, a_1, a_2, b_0, b_1, b_2$ in the general case. But the reasoning above shows that there are only three possibilities for what the third entry could be. In fact, the other two minimums will occupy the first two entries of $M$. So it can be said that the first entry of $M$ is the maximum of $min(a_0, b_0)$, the second entry of $M$ is the maximum of $min(a_1, b_0), min(a_0, b_1)$, and the third entry of $M$ is the maximum of $min(a_2, b_0), min(a_1, b_1), min(a_0, b_2)$. So in general, the $i$th entry of $M$ is the maximum of $min(a_{i - j}, b_j)$for all $j$ from 1 to $i$.
\\\\
The outline for the algorithm is as follows: use $n^2$ processors to calculate all pairwise minimums in constant time, and store the results in a table $A$ such that $A[i][j] = min(a_{i - j}, b_j)$ (unless $j > i$, in which case $A[i][j] = -\infty$, which will never be the maximum).
\\\\
Note also that when $i$ or $j$ exceeds $n$, this implies there are either no more $a$s or no more $b$s to consider, and so the rest of $M$ must be filled with either $b$s or $a$s respectively. And so in these cases no minimum is calculated; the entry in the other list is returned. Consider when $i = 2n$. That implies that the last row is the maximum of $2n$ different entries: $a_0, a_1, ... a_{n - 1}, b_0, b_1, ... b_{n - 1}$. This makes sense intuitively, since the last entry in $M$ must have the maximum value of all of $a$ and $b$.
\\\\
After $A$ gets populated in constant time by $n^2$ processors, $M[i]$ must be filled in with the maximum entry $A[i][j]$ for all $j$. Each row can do this calculation in constant time with $n^2$ processors as discussed in class. Since $n$ rows need to do this, $n^3$ total processors are needed. After a constant number of steps, $M$ will be fully populated with the merged increasing elements of $a$ and $b$, and so the algorithm is finished.

\begin{algorithm}[H]
	\begin{algorithmic}%[1]
		\caption{CRCW Merge two sorted length $n$ lists in $O(1)$}
		\Require Increasing lists $a_1...a_n$, $b_1...b_n$
		\If{$j < i$} \Comment{First, populate each row $i$ of $A$ with the candidate entries that could be in position $i$ of the merged list.}
			\If{$i > n \land j > n$}
				\State $A[i][j] \gets a_{i - j}$
			\ElsIf{$i > n \land j < n$}
				\State $A[i][j] \gets b_j$
			\Else
				\State $A[i][j] \gets min(a_{i - j}, b_j)$
			\EndIf
		\Else
			\State $A[i][j] \gets -\infty$
		\EndIf
		
		
		% TODO: add the maximum calculation.
	\end{algorithmic}
\end{algorithm}

\section*{17}
Break the input into $\sqrt{n}$ chunks of size $\sqrt{n}$ and recursively solve the problem in parallel with $\sqrt{n}$ processors assigned to each chunk. We have $n$ processors, so the $\sqrt{n}$ max values returned by the chunks can be "maxed" in constant time using our $n$ processors (see hint). 
\end{document}
