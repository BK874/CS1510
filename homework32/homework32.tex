
\documentclass[letterpaper,notitlepage,twoside]{article}

% Basic imports, increase margins...
\usepackage[margin=0.75in]{geometry}
% Finite State Machine stuff
\usepackage{pgf}
\usepackage{tikz}
\usetikzlibrary{arrows, automata}
% Format tables nicely
\usepackage[latin1]{inputenc}
\usepackage{array}

\usepackage{amsfonts}
\usepackage{amssymb}
\usepackage{amsmath,amsthm}

\usepackage{algorithm}          %  float wrapper for algorithms.
\usepackage{algpseudocode}      % layout for algorithmicx
\usepackage{float} % http:/ctan.org/pkg/flot - [H] float parameter forces a 'float' (algorithm block) to remain in a location

\renewcommand{\implies}{\Rightarrow} % redefine command "implies"
\renewcommand{\iff}{\Leftrightarrow} % double arrow
\newcommand{\maps}{\rightarrow} % define command "map"
\newcommand{\union}{\cup}
\newcommand{\intersect}{\cap}
\newcommand{\N}{\mathbb{N}} % natural number
\newcommand{\Q}{\mathbb{Q}} % rational number
\newcommand{\R}{\mathbb{R}} % real number
\newcommand{\Z}{\mathbb{Z}} % integers
\newcommand\tab[1][1cm]{\hspace*{#1}} %\tab command

% Add more packages that you use here...

\begin{document}
\title{Homework 32}
\author{Brian Knotten, Brett Schreiber, Brian Falkenstein}
\maketitle

\section*{18}
The outline for this algorithm is as follows: Since the input is a list of
(possibly overlapping) integers $x_1...x_n$ between $1$ and $n$, then each
processor $p_i$ of the $n$ processors can look at its corresponding integer
$x_i$ and confirm that the number $x_i$ is in the input by writing a 1 to an
(originally 0) memory location $IsInInput[i]$. When $IsInInput[i] = 1$, the
integer $i$ is in the input. If no processor writes to $IsInInput[i]$, then it
stays 0, which means that the integer $i$ is not in the input. This stage of
the algorithm takes only constant time, since $n$ processors working on $n$
integers each do one read step and one write step.
\\\\
Next, each processor $p_i$ will read $IsInInput[n - i + 1]$. If $p_i$ finds a 1,
then it will write $n - i + 1$ to the memory location $Maximum$. $Maximum$ will
contain the maximum of $x_1...x_n$ after this step, since the lowest numbered
processor will write the highest valued number. If $p_i$ reads a 1 from
$IsInInput[n - i + 1]$, then it will write $n - i + 1$ to $Maximum$ which
will be the maximum if and only if all $p_j, j < i$ read a 0 in
$IsInInput[n - j + 1]$, which implies there are no higher numbers in the input.
If there is a larger number, then a lower register (with a higher priority) will
write to $Maximum$ instead. This step also takes constant time, because each
processor performs one read and one write.

\begin{algorithm}[H]
  \begin{algorithmic}%[1]
    \caption{CRCW Priority $O(1)$ algorithm for maximum with $n$ processors.}
    \Require Input $x_1...x_n$, an $n$-sized memory location $IsInInput$, and a memory location $Maximum$.
    \State $IsInInput[i] \gets 0$ \Comment{First, zero out the $IsInInput$ array.}
    \State $IsInInput[x_i] \gets 1$ \Comment{This is a CRCW Common step, since all processors write the same number: 1.}
    \If{$IsInInput[n - i + 1] == 1$}
        \State $Maximum \gets n - i + 1$ \Comment{This is the CRCW Priority step.}
    \EndIf
    \If{$i == 0$} \Comment{Designate one processor}
        \State Output $Maximum$ \Comment{to output the maximum}.
    \EndIf
  \end{algorithmic}
\end{algorithm}


\section*{19}
The outline for this algorithm is as follows: Much like 18, each processors $p_i$ of the $n$ processors looks at its corresponding integer $x_i$ and confirms that $x_i$ is in the input by writing a 1 to the (initially 0) memory location $IsInInput[i]$ i.e. when $IsInInput[i] = 1$, the integer $i$ is in the input and when $IsInInput[i] = 0$, the integer $i$ is not in the input. As in 18, this step takes constant time because each processor works on one integer and performs one read step and one write step. \\\\
Next, each processor $p_i$ reads $IsInInput[i]$ and, if $p_i$ finds a 1, it writes a 1 to the array $RootChunks[\lceil \frac{i}{\sqrt{n}} \rceil -1]$, which is an array of size $\sqrt{n}$ initialized to all 0s. If $RootChunks[k] = 1$, then there is a 1 within the $kth$ $\sqrt{n}$-sized chunk of $IsInInput$ i.e. one of the integers from $k*\sqrt{n}$ to $(k+1)*\sqrt{n}$ is present in the input. If $RootChunks[k] = 0$, then there is not a 1 within the $kth$ $\sqrt{n}$-sized chunk of $IsInInput$ i.e. none of the integers from $k*\sqrt{n}$ to $(k+1)*\sqrt{n}$ are present in the input. This step takes constant time because each processor works on one integer and performs one read step and one write step to one of four locations. Each processor writes a 1, if anything, to one of four locations, so it does not violate the CRCW Common restriction. \\\\
Next, the $n$ processors find the max index of $RootChunks$ that contains a 1. That is, we use $n$ processors to find the max of the $\leq \sqrt{n}$ indices that contain a 1. This subroutine is referred to in the algorithm as $IndicesContainingOne$. The result of this step, $m$, corresponds to the $mth$ $\sqrt{n}$-size chunk of $IsInInput$ - this chunk contains the largest value of the input array. \\\\
Finally, the $n$ processors then find the max index of the $mth$ $\sqrt{n}$-size chunk of $IsInInput$ that contains a 1. The resulting index is the max of the input array $x_1, ..., x_n$. Both this step and the prior step take constant time, as for both we are using $n = j^{2}$ processors to compute the max of $\sqrt{n} = j$ integers, which takes $O(1)$ time. \\\\
We have four steps that each take constant time, so this algorithm is $O(1)$ on a CRCW Common PRAM.

\begin{algorithm}[H]
  \begin{algorithmic}%[1]
    \caption{CRCW Common $O(1)$ algorithm for maximum with $n$ processors.}
    \Require Input $x_1...x_n$, an $n$-sized memory location $IsInInput$, a $\sqrt{n}$-sized memory location $RootChunks$, a memory location $MaxIndex$ and a memory location $Maximum$.
    \State $IsInInput[i] \gets 0$ \Comment{First, zero out the $IsInInput$ array.}
    \State $IsInInput[x_i] \gets 1$ \Comment{This is a CRCW Common step, since all processors write the same number: 1.}
    \If{$IsInInput[i] == 1$}
        \State $RootChunks[\lceil \frac{i}{\sqrt{n}} \rceil -1] \gets 1$ \Comment{This is a CRCW Common step.}
    \EndIf
    \State $MaxIndex \gets Max(IndicesContainingOne(RootChunks))$ \Comment{The $Max$ algorithm from problem 17}
    \State $Maximum \gets Max(IndicesContainingOne(IsInInput[MaxIndex*\sqrt{n}]$ to $IsInInput[MaxIndex*\sqrt{n}]))$
  \end{algorithmic}
\end{algorithm}

\section*{20}
Its clear that this algorithm would be incredibly trivial if the machine were not limited to exclusive write. Each processor would simply read its corresponding $B$ value, and write that index in $A$ to its corresponding index in $C$. However, due to the data bottleneck that occurs with $A$, we must make copies. \\
The algorithm begins with each processor making $log^2(n)$ copies of its corresponding index in $A$. These copies are put into a $log^2(n) \times n$ array. This step will take $log^2(n)$ time, as each processor makes $log^2(n)$ copies in sequence. \\
The processors are now broken into groups of size $n/log^2(n)$. This gives us $log^2(n)$ groups. Thus, each grouping of processors can now be assigned its own copy of $A$. Each group could do its writings in sequence, with each processor in the group making its write, and then allowing the next processor to. Since each group has its own copy of $A$, and the groups do their writings sequentially, there is no concurrent reading. 


\end{document}
