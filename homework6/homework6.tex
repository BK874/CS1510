
\documentclass[letterpaper,notitlepage,twoside]{article}

% Basic imports, increase margins...
\usepackage[margin=0.75in]{geometry}
\usepackage{amssymb}
\usepackage{amsmath}

% Finite State Machine stuff
\usepackage{pgf}
\usepackage{tikz}
\usetikzlibrary{arrows,automata}
\usepackage{pgfplots}

% Format tables nicely
\usepackage[latin1]{inputenc}
\usepackage{array}
\usepackage{booktabs}
\setlength{\heavyrulewidth}{1.5pt}
\setlength{\abovetopsep}{4pt}

\usepackage{amsfonts} 
\usepackage{amssymb}
\usepackage{amsmath,amsthm}

\renewcommand{\implies}{\Rightarrow} % redefine command "implies"  
\renewcommand{\iff}{\Leftrightarrow} % double arrow
\newcommand{\maps}{\rightarrow} % define command "map" 
\newcommand{\union}{\cup}
\newcommand{\intersect}{\cap}
\newcommand{\N}{\mathbb{N}} % natural number 
\newcommand{\Q}{\mathbb{Q}} % rational number 
\newcommand{\R}{\mathbb{R}} % real number 
\newcommand{\Z}{\mathbb{Z}} % integers 
\newcommand\tab[1][1cm]{\hspace*{#1}} %\tab command

% Add more packages that you use here..

\title{CS 1510 Homework 6}
\author{Brian Falkenstein, Brian Knotten, Brett Schreiber}

\begin{document}
\maketitle

\section*{1}
\subsection*{a}
Let $Time$ be a function denoting the number of steps required to compute a $T$ on an input. \\
Computing $T(n)$ recursively requires at least the number of steps to calculate $T(n - 1)$ and $T(n - 2)$, as well as a multiplication step. \\
$Time(n) \geq Time(n - 1) + Time(n - 2) + 1$ \\
So computing $T(n)$ takes more steps than computing $T(n - 2)$ twice. \\
$Time(n) \geq 2 * Time(n - 2)$ \\
$Time(n) \geq 4 * Time(n - 4) \geq 8 * Time(n - 6) \geq 16 * Time(n - 8) ...$ \\
$Time(n) \geq 2^{n/2}$ \\
So computing $T(n)$ requires exponentially many operations.

\subsection*{b}
Assuming $T(0), T(1), T(2)...T(n -2), T(n - 1)$ are already computed, it takes $n$ multiplication steps, $T(0) * T(1), T(1) * T(2)...T(n - 2) * T(n - 1)$ and $n$ addition steps to compute $T(n)$, so it takes $2n = O(n)$ steps to do the multiplication and addition. \\
From the bottom up, $T(0), T(1), T(2)...T(n - 1)$ need to be computed, so that is $n$ computations, each of which take $O(n)$ steps. Therefore by starting from the bottom up and caching results, computing $T(n)$ takes $O(n^2)$ steps.

\subsection*{c}
\begin{verbatim}
int T(int n) {
    int T[n + 1];
    T[0] = 2;
    T[1] = 2;
    
    int sum = 0;
    for(int i = 2; i <= n; i++) {
        sum += T[i - 1] * T[i - 2];
        T[i] = sum;
    }
    
    return T[n];
}
\end{verbatim}

\end{document}
