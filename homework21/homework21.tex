\documentclass[letterpaper,notitlepage,twoside]{article}

% Basic imports, increase margins...
\usepackage[margin=0.75in]{geometry}
% Finite State Machine stuff

% Format tables nicely
\usepackage[latin1]{inputenc}
\usepackage{array}

\usepackage{amsfonts}
\usepackage{amssymb}
\usepackage{amsmath,amsthm}

\renewcommand{\implies}{\Rightarrow} % redefine command "implies"
\renewcommand{\iff}{\Leftrightarrow} % double arrow
\newcommand{\maps}{\rightarrow} % define command "map"
\newcommand{\union}{\cup}
\newcommand{\intersect}{\cap}
\newcommand{\N}{\mathbb{N}} % natural number
\newcommand{\Q}{\mathbb{Q}} % rational number
\newcommand{\R}{\mathbb{R}} % real number
\newcommand{\Z}{\mathbb{Z}} % integers
\newcommand\tab[1][1cm]{\hspace*{#1}} %\tab command

% Add more packages that you use here...

\begin{document}
\title{Homework 21}
\author{Brian Knotten, Brett Schreiber, Brian Falkenstein}
\maketitle
\section*{6}

\subsection*{3-Sum $\leq_{n \log n}$ 3-Sum-Part}
3-SumAlgorithm($z_1,...,z_n$)\\
\tab Let $A = \{z_1,...,z_n\}$\\
\tab Let $B = \{z_1,...,z_n\}$\\
\tab Let $C = \{z_1,...,z_n\}$\\
\tab Return 3-Sum-PartAlgorithm($A, B, C$)
\\\\
Since 3-Sum-Part still contains the restriction that $i \neq j \neq k$, then the existence of a 3-Sum all from one set is equivalent to the existence of a 3-Sum of three equal sets. Since the transformation of the input takes $O(n)$ time, and since there is no transformation of the boolean output, the overall algorithm preserves the $O(n \log n)$ runtime.

\subsection*{3-Sum-Part $\leq_{n \log n}$ 3-Sum}
3-Sum-PartAlgorithm($A$, $B$, $C$)\\
\tab Let $A'$ = $A \cdot 2 + 3$; that is, for each element in $A$ we are multiplying by 2 and adding 3.\\
\tab Let $B'$ = $B \cdot 2 + 5$; that is, for each element in $B$ we are multiplying by 2 and adding 5.\\
\tab Let $C'$ = $C \cdot 2 - 8$; that is, for each element in $C$ we are multiplying by 2 and subtracting 8.\\
\tab Let S = \{$A', B', C'$\} = \{$a'_1,...,a'_n, b'_1,...,b'_n, c'_1,...,c'_n$ \}\\
\tab Return 3-SumAlgorithm(S)\\\\
This reduction works assuming all elements in $A$, $B$, $C$ are integers. Because every element in $A$ and $B$ is being doubled and then increased by a positive amount, the sum of any three elements from $A$ or any three elements from $B$ is being increased by a positive amount, preventing the sum from being 0. By the same logic, decreasing every element of $C$ decreases the overall sum of its elements, preventing it from being 0. Because every element is doubled (and is an integer), the edge case of the sum of the elements of one of the collections being 0 after transformation is prevented - ex: $A = \{ -3, -3, -3\}$. \\
Since the transformation takes $O(n)$ time, and there is no transformation of the boolean output, the overall algorithm preserves the $O(n \log n)$ runtime.

\subsection*{3-Sum $\leq_{n \log n}$ 3-Colinear}
This is a reduction we learned in class.
\\\\
3-SumAlgorithm($z_1,...,z_n$):\\
\tab for $i = 1$ to $n$ do:\\
\tab\tab Let $p_i = (z_i, z_i^3)$\\
\tab Return 3-ColinearAlgorithm($p_1,...,p_n$)
\\\\
As discussed in class,
\begin{align*}
a + b + c = 0 &\iff \frac{b^3 - a^3}{b - a} &= \frac{c^3 - b^3}{c - b}\\
&\iff \frac{(b - a)(b^2 + ba + a^2)}{b - a} &= \frac{(c - b)(c^2 + cb + b^2)}{c - b}\\
&\iff b^2 + ba + a^2 &= c^2 + cb + b^2\\
&\iff b^2 + ba + a^2 - c^2 - cb - b^2 &= 0\\
&\iff ba + a^2 - c^2 - cb &= 0\\
&\iff b(a - c) + (a - c)(a + c) &= 0\\
&\iff (a - c)(a + b + c) &= 0\\
&\iff a + b + c &= 0
\end{align*}

\subsection*{3-Colinear-Line $\leq_{n \log n}$  3-Colinear}
An instance of 3-Colinear-Line can be thought of as a special case of an instance of 3-Colinear with an extra restriction of not being true when the only colinear line in the input has a slope of 0. So the algorithm is as follows:
\\\\
3-Colinear-LineAlgorithm($p_1,...,p_n$)
\tab Return 3-Colinear-Line($p_1,...,p_n$) where horizontal lines have been filtered out.
\\\\
3-Colinear-LineAlgorithm retains $O(n \log n)$ runtime because there is no transformation of input to output.

\subsection{3-Colinear-Line $\leq_{n \log n}$ 3-Sum}
If an instance of 3-Colinear-Line is true, then it must have three points $(a, 0) (b, 1), (c, 2)$, which are colinear, since this problem does not allow horizontal lines as solutions. Therefore the slope between $a$ and $b$ must be the same as the slope between $b$ and $c$ as follows:
\\\\
\begin{align*}
\frac{2 - 1}{c - b} &= \frac{1 - 0}{b - a}\\
\frac{1}{c - b} &= \frac{1}{b - a}\\
c - b &= b - a\\
a - 2b + c &= 0
\end{align*}

So there must exist three numbers $a$, $-2b$, and $c$ in an instance of 3-Sum which makes it true. So try manipulating each possible $b$ in the 3-Colinear-Line instance to get an output for 3-SumAlgorithm. So the algorithm for 3-Colinear-Line is as follows:
\\\\
3-Colinear-LineAlgorithm($p_1,...,p_n$):\\
\tab For $i = 1$ to $n$ do:\\
\tab\tab Let $(x_i, y_i) = p_i$ for $i = 1$ to $n$.\\
\tab\tab If 3-SumAlgorithm($x_1,...-2x_i,...x_n$) then return True\\
\tab Otherwise return false.

\section*{7}
\end{document}
