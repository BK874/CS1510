\documentclass[letterpaper,notitlepage,twoside]{article}

% Basic imports, increase margins...
\usepackage[margin=0.75in]{geometry}
% Finite State Machine stuff

% Format tables nicely
\usepackage[latin1]{inputenc}
\usepackage{array}

\usepackage{amsfonts}
\usepackage{amssymb}
\usepackage{amsmath,amsthm}

\renewcommand{\implies}{\Rightarrow} % redefine command "implies"
\renewcommand{\iff}{\Leftrightarrow} % double arrow
\newcommand{\maps}{\rightarrow} % define command "map"
\newcommand{\union}{\cup}
\newcommand{\intersect}{\cap}
\newcommand{\N}{\mathbb{N}} % natural number
\newcommand{\Q}{\mathbb{Q}} % rational number
\newcommand{\R}{\mathbb{R}} % real number
\newcommand{\Z}{\mathbb{Z}} % integers
\newcommand\tab[1][1cm]{\hspace*{#1}} %\tab command

% Add more packages that you use here...

\begin{document}
\title{Homework 20}
\author{Brian Knotten, Brett Schreiber, Brian Falkenstein}
\maketitle
\section*{3}
To prove that if there is an $O(n^k)$ algorithm for squaring an degree $n$ polynomial (call this $Sqare(C)$, then there is an $O(n^k)$ algorithm for multiplying 2 degree $n$ polynomials (call this $Mult(A, B)$, we must transform the input of $Mult(A, B)$ to the input of $Square(C)$ in polynomial time, and then transform the output of $Square(C)$ to the output of $Mult(A, B)$, also in polynomial time. \\
Given as input to $Mult(A, B)$ two degree $n$ polynomials $A$ and $B$, we can construct a polynomial $C$ by doing the following: 
\begin{itemize}
\item Initialize the solution polynomial $P$ to 0, so $P=0$. 
\item Let $j$ be the number of coefficient variable pairs in $A$ and $k$ be the number of pairs in $B$. 
\item For $1<=i<=j$, take the $i'th$ coefficient variable pair from $A$ and add it to $B$, so $C_i$ = $c_ix_i + B$
\item Pass $C_i$ into $Square(C)$.
\item From the output of $Square(C_i)$, add the $c_2x_2 ... c_{k+1}x_{k+1}$ coefficient variable pairs to $P$. 
\item Repeat for all $i's$ up to $j$ (all pairs in $A$). 
\item Return $P$. 
\end{itemize}
Note that the code for $Mult(A, B)$ will call $Square(C)$ $j$ times, although since $j$ is a constant, and we are assuming $Square(C)$ has an $O(n^k)$ algorithm, the running time for calling $Square(C)$ will still be polynomial (constant times polynomial). The transformation of $A$ into $C_i$ takes constant time, as we just add a term from $A$ to $C_i$. The transformation of the output also takes poly time, as we are simply grabbing $k+1$ terms from the output of $Square(C)$ and adding them to $P$, and repeating this a constant number of times. \\
As an example, consider the two polynomials $A=a^2 + b$ and $B = c^2 + d$. This algorithm will construct $C_1 = a^2 + c^2 + d$. The output of $Square(C_1)$ will then be $a^4 + a^2c^2 + a^2d...$ (we don't care about the rest of the output). We ignore the first term ($a^4$) and add the terms $a^2c^2$ and $a^2d$ to $P$, so now $P = a^2c^2 + a^2d$. Now, construct $C_2 = b + c^2 + d$. Squaring $C_2$ gives us $b^2 + bc^2 + bd ...$. We then add $bc^2$ and $bd$ to $P$, giving us $P=a^2c^2 + a^2d + bc^2 + bd$, the solution to $A\times B$. 
\section*{5}
To prove that if an algorithm for one of Undirected Graph Isomorphism, Directed Graph Isomorphism, and Mixed Graph Isomorphism implies that they all do, the following reductions need to be made:

\subsection*{Directed Graph Isomorphism $\leq_p$ Mixed Graph Isomorphism}
Assume there exists an algorithm for Mixed Graph Isomorphism called $MISO$. Then it is possible to construct a poly-time algorithm for Directed Graph Isomorphism called $DISO$ as follows:
\\\\
$DISO(G, H):$\\
\tab return $MISO(G, H)$
\\\\
A purely directed graph can be thought of as a special case of a mixed graph, so an algorithm for Mixed Graph Isomorphism works just as well on inputs that only have directed edges. Moreover, since $MISO$ is poly-time, and since $DISO$ makes no transformations from input to input or from output to output, then $DISO$ is also poly-time.

\subsection*{Undirected Graph Isomorphism $\leq_p$ Mixed Graph Isomorphism}
Assume there exists an algorithm for Mixed Graph Isomorphism called $MISO$. Then it is possible to construct a poly-time algorithm for Undirected Graph Isomorphism called $UISO$ as follows:
\\\\
$UISO(G, H):$\\
\tab return $MISO(G, H)$
\\\\
A purely undirected graph can be thought of as a special case of a mixed graph, so an algorithm for Mixed Graph Isomorphism works just as well on inputs that only have directed edges. Moreover, since $MISO$ is poly-time, and since $UISO$ makes no transformations from input to input or from output to output, then $UISO$ is also poly-time.

\subsection*{Undirected Graph Isomorphism $\leq_p$ Directed Graph Isomorphism}
Assume there exists an algorithm for Directed Graph Isomorphism called $DISO$. Then it is possible to construct a poly-time algorithm for Undirected Graph Isomorphism called $UISO$ as follows:
\\\\
$UISO(G, H):$\\
\tab Let $G' = Directed(G)$\\
\tab Let $H' = Directed(H)$\\
\tab return $DISO(G', H')$
\\\\
Where $Directed$ is a supplementary function as follows:
$Directed(G):$\\
\tab for each undirected edge $e$ between $u$ and $v$ in $G$:\\
\tab\tab add a directed edge from $u$ to $v$ into $G'$ and\\
\tab\tab add a directed edge from $v$ to $u$ in $G'$\\
\tab return $G'$
\\\\
An undirected graph can be thought of as a special case of a directed graph where each undirected edge corresponds to two mutual directed edges. Since two directed graphs can only be isomorphic if they both share these pairs of mutual directed edges, the algorithm remains correct after the input has been transformed. Since this transformation is polynomial in the number of edges, and since the output does not need to be transformed, then $UISO$ is also a poly-time algorithm.
\end{document}
