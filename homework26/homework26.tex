\documentclass[letterpaper,notitlepage,twoside]{article}

% Basic imports, increase margins...
\usepackage[margin=0.75in]{geometry}
% Finite State Machine stuff
\usepackage{pgf}
\usepackage{tikz}
\usetikzlibrary{arrows, automata}
% Format tables nicely
\usepackage[latin1]{inputenc}
\usepackage{array}

\usepackage{amsfonts}
\usepackage{amssymb}
\usepackage{amsmath,amsthm}

\renewcommand{\implies}{\Rightarrow} % redefine command "implies"
\renewcommand{\iff}{\Leftrightarrow} % double arrow
\newcommand{\maps}{\rightarrow} % define command "map"
\newcommand{\union}{\cup}
\newcommand{\intersect}{\cap}
\newcommand{\N}{\mathbb{N}} % natural number
\newcommand{\Q}{\mathbb{Q}} % rational number
\newcommand{\R}{\mathbb{R}} % real number
\newcommand{\Z}{\mathbb{Z}} % integers
\newcommand\tab[1][1cm]{\hspace*{#1}} %\tab command

% Add more packages that you use here...

\begin{document}
\title{Homework 26}
\author{Brian Knotten, Brett Schreiber, Brian Falkenstein}
\maketitle
\subsection*{22}


\subsection*{23}


\subsection*{24}
We will reduce the Vertex Cover problem to the Fox, Goose, and Bag of beans puzzle, hereafter referred to as the FGB problem. In order to do so, we will create a graph $H$ and integer $\ell$ that that represent a boat of size $\ell$ that can safely transport the objects represented by the vertices of $H$ iff there exists a vertex cover of size $k$ for the graph $G$. \\
In order to construct $H$, copy all of the vertices and edges from $G$ and add two additional vertices $x$ and $y$ that are disconnected from all other vertices except each other. $\ell$ is merely $k+1$. \\
This transformation is clearly polynomial in the size of $G$, as $H$ is constructed using the size of $G$ + a constant number of vertices and edges. \\
To prove $G$ has a vertex cover of size $k$ iff there exists a boat of size $\ell$ that can safely transport the objects represented by the vertices of $H$, we must prove the relationship both ways: \\\\
First, assume there exists a boat of size $\ell$ that can safely transport the objects represented by the vertices of $H$. If so, then consider an arbitrary trip using the boat. Because there is an edge between $x$ and $y$, then it must be the case that $x$ and $y$ are not both on the boat or both on one of the shores. The vertices remaining on each shore must be safe to be together i.e. must both be an independent set. However, if this is the first trip across, then one of $x$ or $y$ must be on the boat. Therefore there are at most $\ell - 1 = k$ of $G$'s vertices on the boat and both the vertices on the boat and the vertices on the shore are an independent set. Thus, the $\leq k$ vertices on the boat during the first trip are a vertex cover of $G$. \\\\ 
Second, assume $G$ has a vertex cover of size $k$. Then a boat of size $k+1=\ell$ can safely carry the items represented by $H$ by putting the vertex cover of $G$ and the new vertex $x$ in the boat. Because a vertex cover touches every edge of a graph, the vertices left on the shore (including $y$) are an independent set of $H$ and can therefore remain on the shore safely. We can then deposit $x$ on the second shore. The vertices in the vertex cover of $G$ must remain on the boat as the vertices on the first shore (i.e. those not in the vertex cover) will be transported across one at a time, so we cannot leave a vertex in the vertex cover on the second shore with one not in the vertex cover. We cannot simply deposit the vertex cover with $x$ and then transport the remaining vertices in one trip, as the number of vertices not in the vertex cover + $y$ may be greater than $k+1$. Therefore we will deposit $x$ first, then transport the vertices not in the vertex cover one by one, and then transport $y$ last and all the vertices in $H$ have been safely transported by a boat of size $\ell = k+1$.\\\\
Therefore, given a poly-time algorithm that solves the FGB problem, we can solve Vertex Cover in polynomial time by using the poly-time transformation given above, a call to the poly FGB algorithm, and outputting 1 if the poly-time algorithm does, and 0 otherwise


\end{document}