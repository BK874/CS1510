\documentclass[letterpaper,notitlepage,twoside]{article}

% Basic imports, increase margins...
\usepackage[margin=0.75in]{geometry}
% Finite State Machine stuff
\usepackage{pgf}
\usepackage{tikz}
\usetikzlibrary{arrows, automata}
% Format tables nicely
\usepackage[latin1]{inputenc}
\usepackage{array}

\usepackage{amsfonts}
\usepackage{amssymb}
\usepackage{amsmath,amsthm}

\usepackage{algorithm}          %  float wrapper for algorithms.
\usepackage{algpseudocode}      % layout for algorithmicx
\usepackage{float} % http:/ctan.org/pkg/flot - [H] float parameter forces a 'float' (algorithm block) to remain in a location

\renewcommand{\implies}{\Rightarrow} % redefine command "implies"
\renewcommand{\iff}{\Leftrightarrow} % double arrow
\newcommand{\maps}{\rightarrow} % define command "map"
\newcommand{\union}{\cup}
\newcommand{\intersect}{\cap}
\newcommand{\N}{\mathbb{N}} % natural number
\newcommand{\Q}{\mathbb{Q}} % rational number
\newcommand{\R}{\mathbb{R}} % real number
\newcommand{\Z}{\mathbb{Z}} % integers
\newcommand\tab[1][1cm]{\hspace*{#1}} %\tab command

% Add more packages that you use here...

\begin{document}
\title{Homework 33}
\author{Brian Knotten, Brett Schreiber, Brian Falkenstein}
\maketitle

\section*{21}
Consider a sequential algorithm for this problem on a binary tree $T$ as follows:
\begin{enumerate}
\item Let $i \gets 1$.
\item Perform an Eulerian tour on the tree.
\item If the node is a leaf, label the node with $i$ and increment $i$.
\end{enumerate}
The result is a tree with each leaf node labelled in-order.
\\
The outline for the parallel version of this algorithm is as follows:
\begin{enumerate}
\item Perform an Eulerian tour on $T$ to return a $3n$ linked-list $L$.
\item Let the nodes in $L$ representing a leaves in the tree (only one visit on the tour) be set to $1$ and let their corresponding processor keep a pointer to its node.
\item Let all other nodes in $L$ be set to $0$.
\item Perform the linked-list parallel prefix algorithm on $L$.
\item Each processor corresponding to a leaf node in $T$ looks at its pointer to the node in $L$ to see a prefix sum. This number represents the number of the leaf in $T$ in an in-order traversal.
\item Have each processor label their leaf node in $T$ with this number.
\end{enumerate}

This algorithm takes $O(\log(n))$ time, since performing an Eulerian tour takes a constant number of steps with $n$ processors, and performing the parallel prefix sum algorithm takes $O(\log(n))$ steps. This algorithm is EREW, because the Eulerian tour algorithm is EREW, the parallel prefix algorithm is EREW, and each processor reads and writes exclusively to its corresponding nodes in the steps particular to this algorithm.
\\\\
Formally, the algorithm is as follows:
\begin{algorithm}[H]
  \begin{algorithmic}%[1]
    \caption{EREW $O(\log(n))$ algorithm for In-Order labeling of leaf nodes with $n$ processors.}
    \Require An $n$-sized binary tree $T$.
	\State $t_i \gets $ the node in $T$ which $p_i$ points to.
	\State $t_p \gets $ $t_i$'s parent node.
	\If{$t_i$ is a leaf node}
		\State Allocate a copy of $t_i$ for creating $L$: $l_{i_1}$
		\If{$t_i$ is a left child}
			\State Link $l_{i_1} \rightarrow l_{p_2}$
		\ElsIf{$t_i$ is a right child}
			\State Link $l_{i_1} \rightarrow l_{p_3}$
		\EndIf
	\Else
		\State $t_l \gets $ $t_i$'s left child.
		\State $t_r \gets $ $t_i$'s right child.
		\State Allocate three copies of $t_i$ for creating $L$: $l_{i_1}$, $l_{i_2}$, $l_{i_3}$.
		\State Link $l_{i_1} \rightarrow l_{l_1}$
		\State Link $l_{i_2} \rightarrow l_{r_1}$
		\If{$t_i$ is a left child}
			\State Link $l_{i_3} \rightarrow l_{p_2}$
		\ElsIf{$t_i$ is a right child}
			\State Link $l_{i_3} \rightarrow l_{p_3}$
		\EndIf
	\EndIf
	\State $L' \gets$ the parallel prefix algorithm on $L$.
	\If{$t_i$ is a leaf node}
		\State Label $t_i$ with $l_{i_1}'$
	\EndIf
  \end{algorithmic}
\end{algorithm}

\section*{23}
As indicated by the hint, to solve the problem of evaluating a binary expression tree with the four standard arithmetic operations as nodes and integers as leaves it will suffice to find a collection of functions that is closed under the four standard operations with constants, composition, and contains the identity and constant functions and apply that class of functions to the algorithm presented in class. \\\\
Note that in class we used the class of functions of the form $ax+b$ (where $a$ is 1 or -1, and $b$ is any number) for addition and subtraction. \\\\
Now we present a new class of functions of the form $\frac{ax+b}{nx+m}$ and show that this class of functions satisfies the presented requirements:
\begin{enumerate}
  \item Identity function: $\frac{1x+0}{0x+1} = \frac{x}{1} = x$
  \item Constant function: $\frac{0x+k}{0x+1} = \frac{k}{1} = k$ where $k$ is any constant
  \item Closed under addition of a constant: $\frac{ax+b}{nx+m} + \ell = \frac{ax+b}{nx+m} + \frac{\ell (nx+m)}{nx+m} = \frac{ax+b+\ell nx+\ell m}{nx+m} = \frac{(a+\ell n)x+b+\ell m}{nx+m} = \frac{yx+z}{nx+m}$ where $y = a + \ell n$ and $z = b + \ell m$
  \item Close under subtraction of a constant: $\frac{ax+b}{nx+m} - \ell = \frac{ax+b}{nx+m} + \frac{(-\ell) (nx+m)}{nx+m} = \frac{ax+b+(-\ell) nx+(-\ell) m}{nx+m} = \frac{(a+(-\ell) n)x+b+(-\ell) m}{nx+m} = \frac{yx+z}{nx+m}$ where $y = a + (-\ell) n$ and $z = b + (-\ell) m$
  \item Closed under multiplication by a constant: $\frac{ax+b}{nx+m} \cdot c = \frac{cax+cb}{nx+m} = \frac{yx+z}{nx+m}$ where $y=ca$ and $z=cb$
  \item Closed under division by a constant: $\frac{\frac{ax+b}{nx+m}}{d} = \frac{ax+b}{nx+m} \cdot \frac{1}{d} = \frac{ax+b}{dnx+dm}  = \frac{ax+b}{yx+z}$ where $y = dn$ and $z = dm$
  \item Closed under the division of a constant: $\frac{d}{\frac{ax+b}{nx+m}} = d \cdot \frac{nx+m}{ax+b} = \frac{dnx+dm}{ax+b} = \frac{yx+z}{ax+b}$ where $y = nx$ and $z = dm$
  \item Closed under the composition of functions: Let $f(x) = \frac{ax+b}{nx+m}$ and let $g(x) = \frac{cx+d}{ex+f}$. Then $f(g(x)) = \frac{a(\frac{cx+d}{ex+f})+b}{n(\frac{cx+d}{ex+f})+m} = \frac{\frac{acx+ad}{ex+f}+b}{\frac{ncx+nd}{ex+f}+m} = \frac{\frac{acx+ad}{ex+f}+\frac{bex+bf}{ex+f}}{\frac{ncx+nd}{ex+f}+\frac{mex+mf}{ex+f}} = \frac{ex+f}{ex+f} \cdot \frac{(acx+ad)+(bex+bf)}{(ncx+nd)+(mex+mf)} = \frac{(ac+be)x+(ad+bf)}{(nc+me)x+(nd+mf)} = \frac{yx+z}{wx+v}$ where $y = (ac+be)$, $z = (ad+bf)$, $w = (nc+me)$, and $v = (nd + mf)$
\end{enumerate}

We have shown that our class of functions is closed under the four standard operations with constants and composition, and contains the identity and constant functions. Therefore, our class of functions can be applied to the $O(log^2n)$ algorithm given in class that applies "cuts" using these functions to every odd numbered left child that is a leaf or even numbered right child that is a leaf at each step. Because our operations can all be done in constant time, our functions can be applied to the class algorithm without changing the $O(log^2n$ runtime on a CREW PRAM with $n$ processors.

\end{document}
