
\documentclass[letterpaper,notitlepage,twoside]{article}

% Basic imports, increase margins...
\usepackage[margin=0.75in]{geometry}
% Finite State Machine stuff
\usepackage{pgf}
\usepackage{tikz}
\usetikzlibrary{arrows, automata}
% Format tables nicely
\usepackage[latin1]{inputenc}
\usepackage{array}

\usepackage{amsfonts}
\usepackage{amssymb}
\usepackage{amsmath,amsthm}

\usepackage{algorithm}          %  float wrapper for algorithms.
\usepackage{algpseudocode}      % layout for algorithmicx

\renewcommand{\implies}{\Rightarrow} % redefine command "implies"
\renewcommand{\iff}{\Leftrightarrow} % double arrow
\newcommand{\maps}{\rightarrow} % define command "map"
\newcommand{\union}{\cup}
\newcommand{\intersect}{\cap}
\newcommand{\N}{\mathbb{N}} % natural number
\newcommand{\Q}{\mathbb{Q}} % rational number
\newcommand{\R}{\mathbb{R}} % real number
\newcommand{\Z}{\mathbb{Z}} % integers
\newcommand\tab[1][1cm]{\hspace*{#1}} %\tab command

% Add more packages that you use here...

\begin{document}
\title{Homework 29}
\author{Brian Knotten, Brett Schreiber, Brian Falkenstein}
\maketitle

\section*{9}
The general outline of the algorithm is as follows:\\\\
Due to the EREW restriction, the algorithm must first make copies of the input string $C$. The algorithm uses $n^2$ processors to make $n^2$ copies of the string in $log(n)$ time. It does this by using $x*n$ processors to make $x$ copies of the input string, where each processor reads a character from the string and writes it to an array. The array is of size $n\times n^2$, each row being one copy of $C$. The array will also be referred to as $C$, as it can be thought of as adding another dimension to the input string $C$, where each added row is a copy of $C$. Note that when $x>n$, we cannot make $xn$ copies in one step, as $x*n > n^2$. However, the additional $((x*n) - n^2)$ copies can be done in constant time. This makes the whole initial copying take $log(n)$. \\\\
Next, $k$ processors are used to write an answer array, we'll call it $M$, where the $i'th$ index of $M$ is $i$ if there exists a prefix and suffix of length $i$, or $0$ otherwise. \\\\
Then, $n^2$ processors are used to check all possible prefix/suffixes of lengths 1 to $k$. Each processor is identified by 2 numbers $i, j$. If its found that there is no prefix suffix of length $k=i$, write a zero to $M[i]$. \\\\
Next, the max of the $M$ array must be found. Because $M[i] = i$ iff a prefix and suffix exist of length $i$, and $0$ otherwise, if we find the max of $M$, we will find the max length of prefix/suffix, and thus $k$. \\\\
Below is the algorithm for a single processor $i, j$:
\begin{algorithm}
    \begin{algorithmic}%[1]
       \caption{EREW $O(log(n))$ algorithm}
        \Require A string $C$ of size $n$ to be expanded to an $n^2\times n$ array of copies, a processor $p_{i, j}$, a memory location $M$ of size $n^2$
        \State $M[i][j] \gets i$ \Comment{Use $n^2$ processors to write $1...k$ into $M[i]$ for all $i$.}
        \State $number\_of\_copies \gets 1$ \Comment{This is a variable to store the current number of copies made.}
        \While{$number\_of\_copies < n^2$}
            \If{$j < number\_of\_copies$} \Comment{Only use the processors needed to make $c$ copies.}
                \State $C[i][j+number\_of\_copies] \gets C[i][j]$ \Comment Copy current character to new copy location
            \EndIf
            \If{$number\_of\_copies > n$}
                \State $number\_of\_copies \gets number\_of\_copies + n$ \Comment{All $n^2$ processors can copy at most $n$ strings eath of length $n$.}
            \Else
                \State $number\_of\_copies \gets number\_of\_copies * 2$ \Comment{Otherwise, with less than $n^2$ processors, the number of copies can be doubled.}
            \EndIf
        \EndWhile
        \If{$C[j][2i+j] \neq C[n - i + j][2i + j]$} \Comment{Each processor compares two individual characters to see if the prefix is of size $k$. The extra dimension on $C$ is to ensure that all processors are reading from different places in memory.}
            \State $M[i][j] \gets 0$ \Comment{If any of the pairs of characters don't match, then that $k$ isn't viable.}
        \EndIf
        
        \State $z \gets \lfloor{n/2}\rfloor$ \Comment{Flatten the $M$ array so that any zero entry makes $M[i][1] \gets 0$.}
        \While{$j < z$}
            \State $M[i][j] \gets MIN(M[i][j], M[i][j + z])$ \Comment{If any entry in the $j$ column is zero, then $M[i][1]$ is 0.}
            \State $z \gets \lfloor{z/2}\rfloor$
        \EndWhile
        
        \State $y \gets \lfloor{n/2}\rfloor$ \Comment{Now get the max of the $M[1]$ array}
        \While {$i < y$}
            \State $M[i][1] \gets MAX(M[i][1], M[i+y][1]$) \Comment{A processor can take the max of 2 values in constant time. Overwrite the greater number into $M[i]$. After $\log n$ iterations, $M[1][1]$ will contain the maximum $k$.}
            \State $y \gets \lfloor{y/2}\rfloor$
        \EndWhile
        \If{$i==1$ and $j==1$} \Comment{Designate the first processor to exclusively read and output the solution.}
            \State Output $M[1][1]$
        \EndIf
    \end{algorithmic}
\end{algorithm}

This algorithm takes $\log n$ time. The first phase of making $n^2$ copies of the input string takes $\log n$ time, since $n^2$ processors can double the number of copies at each step, and so can reach $n^2$ copies in $\log n$ steps. The second phase of using $n^2$ processors to check the equality of the prefixes and suffixes takes constant time. The third phase checks to see if any cell in $M$ is zero, in which case the whole row (represented by $M[i][1]$ becomes zero. This takes $\log n$ time since $z$ is iteratively halved. The fourth phase takes the maximum over $M[i][1]$ so that $M[1][1]$ contains the maximum $k$ and outputs. This takes $\log n$ time since $y$ is iteratively halved.

\section*{10}
The general outline for this algorithm is as follows:
\\\\
At the first time step, use $n - 1$ processors to write the numbers $1, 2, 3, 4, ... n - 1$ in memory location $M$. These represent values for $k$ such that there exists a matching prefix and suffix of size $k$.
\\\\
At the second time step, use $n^2$ processors to compare the prefixes and suffixes of all lengths for $k$ from $1$ to $n - 1$. If the prefix and suffix aren't equal to each other, then "zero out" the location in memory. For example, if the first 3 characters do not match the last 3 characters, then the numbers in $M$ become $1, 2, 0, 4, ... n -1$.
\\\\
Finally, at the third time step, use $n^2$ processors to find the maximum number left in $M$. This will return the maximum valid $k$, and it can be done in constant time as discussed in class.
\\\\
Below is an algorithm for one of the $(n - 1)^2$ processors, $i, j$. (Without loss of generality, the processors are labelled with two numbers, $i \in [1...n-1]$ and $j \in [1...n-1]$ for easier usage).

\begin{algorithm}
    \begin{algorithmic}%[1]
        \caption{CRCW Common $O(1)$ algorithm}
        \Require A string $C$ of size $n$, a processor $p_{i, j}$, a memory location $M$ of size $n - 1$ and a memory location $And$ of size $n - 1$.
        \State $M[i] \gets i$ \Comment{Use $n$ processors to copy the numbers $1, 2, 3 ... n - 1$ into $M$.}
        \If{$C[j] \neq C[n - i + j]$}
            \State $M[i] \gets 0$ \Comment{If any of the pairs of characters don't match, then that $k$ isn't viable.}
        \EndIf
        \State $And[i] \gets 1$ \Comment{Perform an EREW AND operation to find a row in $T$ of all 1s.}
        \If{$M[i] < M[j]$} \Comment{Perform all possible pairwise comparisons of $M$ using $n^2$ processors.}
            \State $And[i] \gets 0$ \Comment{If $M[i]$ is less than any $M[j]$, then $M[i]$ cannot be the maximum $k$.}
        \EndIf
        \If{$And[i] = 1$}
            \State Output $i$ \Comment{That row is the maximum $k$. A maximum always exists, so this will always output.}
        \EndIf
    \end{algorithmic}
\end{algorithm}

This algorithm runs in $O(1)$ time, since each processor only performs a constant number of operations, as described above.

\section*{11}
\subsection*{a}
Let the input to the algorithm be two $n$-bit integers $X = x_0x_1...x_{n-1}$ and $Y = y_0y_1...y_{n-1}$. Let $p$ denote the number of processors and $p = n = |X| = |Y|$. For ease of outlining the algorithm, without loss of generality let $n = 2^k$ for some $k \geq 1$. Additionally, note that any $n$-bit integer $Z$ can be easily split into the sum of two integers $Z_1$ and $Z_2$ where $Z_1$ is an $n$-bit integer and $Z_2$ is a $k$-bit integer if we let $Z_2$ be the first (counting from right to left) $k$ bits of $Z$ and let $Z_1$ be the remaining $n-k$ bits multiplied by $2^{k}$. For example, let $n = 8$, $Z = 11101011$ and $k = 3$. Then $Z_2= 011$, $Z_1 = 11101 * 2^{3} = 11101000$, and $Z = Z_1 + Z_2 = 11101000 + 011 = 11101011$. Additionally, note that the construction of $Z_2$ is merely the concatenation of $k$ zeros onto the right side of the remaining $n-k$ bits. 
\\\\
The general outline for this algorithm is as follows:
\\\\
Using the Divide and Conquer methodology, the inputs $X$ and $Y$ are split into two chunks, $X_1, X_2$ and $Y_1, Y_2$ constructed as described above, where $X_1$ and $Y_1$ are the last (from right to left) $n/2$ digits of $X$ and $Y$ (respectively) multiplied by $2^{n/2}$ and $X_2$ and $Y_2$ are the first (from right to left) $n/2$ digits of $X$ and $Y$ respectively. The pairs $X_1$ and $Y_1$ are then summed together recursively, as are $X_2$ and $Y_2$. If the sum of $X_2$ and $Y_2$ have a carry, then 
\subsection*{b}
The EREW PRAM algorithm works similarly to the CREW PRAM algorithm in part a, except at each of the $log(n)$ steps copies of the $n$ bits must be made in $log(n)$ time, so the algorithm runs in $O(log^{2} n)$ time.


\end{document}
