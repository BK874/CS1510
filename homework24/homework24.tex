\documentclass[letterpaper,notitlepage,twoside]{article}

% Basic imports, increase margins...
\usepackage[margin=0.75in]{geometry}
% Finite State Machine stuff

% Format tables nicely
\usepackage[latin1]{inputenc}
\usepackage{array}

\usepackage{amsfonts}
\usepackage{amssymb}
\usepackage{amsmath,amsthm}

\renewcommand{\implies}{\Rightarrow} % redefine command "implies"
\renewcommand{\iff}{\Leftrightarrow} % double arrow
\newcommand{\maps}{\rightarrow} % define command "map"
\newcommand{\union}{\cup}
\newcommand{\intersect}{\cap}
\newcommand{\N}{\mathbb{N}} % natural number
\newcommand{\Q}{\mathbb{Q}} % rational number
\newcommand{\R}{\mathbb{R}} % real number
\newcommand{\Z}{\mathbb{Z}} % integers
\newcommand\tab[1][1cm]{\hspace*{#1}} %\tab command

% Add more packages that you use here...

\begin{document}
\title{Homework 24}
\author{Brian Knotten, Brett Schreiber, Brian Falkenstein}
\maketitle
\subsection*{15}
Define the problem mentioned as $I-SAT$ (for inequality satisfiability). Show that CNFSAT $\leq _{poly}$ I-SAT. That is, show that we can construct an instance of I-SAT that is satisfiable if and only if the instance of CNF-SAT that we construct it from is also satisfiable. \\
For each literal $x$ in CNF-SAT, construct 2 inequalities as follows:\\
$$x + \overline x + x\overline x \leq 1$$
$$x + \overline x + x\overline x \geq 1$$
This forces every literal $x$ and its negation $\overline x$ to have one assigned to 1, and one to 0. If both were assigned to  0, it would violate the second inequality. If both were assigned to 1, it would violate the first inequality. If either were assigned to any value other than 0 or 1, it would violate one of the rules. Thus, every literal must be assigned a value of either 0 or 1, and both a literal and its negation may not have the same assignment. Take an assigned value of 1 to mean that literal is true in the CNF solution. \\
For each CNF-SAT clause, construct an inequality by adding the literals in the clause together, and setting it greater than or equal to 1. For example, for the clause $(x \vee \overline y \vee  z)$:
$$x + \overline y + z \geq q$$
This forces at least one literal per clause to be true, while allowing for more than one to be. Doing this for every literal and clause will yield the I-SAT instance. Note that the transformation is in poly time with respect to the number of literals in the original CNFSAT instance (constant time to construct 2 inequalities for each literal) and the number of clauses (constant in the number of literals in each clause). 

\subsection*{18}
First, we will show that a similar problem that we are calling Fixed-Vertex Triad-Free Set is NP-Hard. Fixed Vertex Triad-Free Subset is a similar problem to finding a triad-free set in a graph $G$, except that one vertex $v \in G$ must be included in the solution set $S$.
\\
Independent Set $\leq_p$ Fixed-Vertex Tried-Free Set. Given an input graph $G$ and an integer $k$, construct a graph $G' = G + v$, where $v$ is connected to every other vertex in $G'$. Call the Fixed Vertex Triad-Free Set algorithm on $G'$ and $k + 1$, and let the fixed vertex be $v$. The algorithm will return true if and only if there exists an independent set of size $k$ in $G$.
\\
This is because for every edge $(v_i, v_j) \in G$, there is a corresponding set of edges that forms a triangle $\{(v_i, v_j), (v_i, v), (v_j, v)\} \in G'$. Since $v$ must be in $S$, that means only one of $v_i$ and $v_j$ can be picked. If they were both picked, it would create a triad.
\\
If $G$ has an independent set of size $k$, then $G'$ has a fixed-vertex triad-free set of size $k + 1$. Add the fixed-vertex $v$ to $G$'s independent set to get a fixed-vertex triad-free set of size $k + 1$ for $G'$. Conversely, if $G'$ has a fixed-vertex triad-free set of size $k + 1$, then $G$ has an independent set of size $k$. Simply remove the fixed-vertex from $G'$'s solution set to get an independent set of size $k$ for $G$.
\\
Now we must show that Triad-Free Set is NP-Hard. Fixed-Vertex Triad-Free Set $\leq_p Triad-Free Set$. The algorithm is as follows:
\\\\
FixedVertexTriadFreeSetAlgorithm($G, k, v$):\\
\tab Let $S = $ TriadFreeSet($G, k$)
\tab if there does not exist a set, return false. It cannot be the case that there exists a solution to our stricter problem if there is not a solution to the more general problem.\\
\tab If $v \in S$, return $S$, since the solution has $v$ in it.\\
\tab Otherwise, $v \notin S$. \\
\tab If $v$ is not a part of any triangles, then return a solution set $S'$ which includes $v$ and excludes some other arbitrary vertex. It is a valid solution since $v$ cannot inadvertently create any triads.\\
\tab If $v$ is a part of some triangles, then...

\end{document}
