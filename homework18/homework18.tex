\documentclass[letterpaper,notitlepage,twoside]{article}

% Basic imports, increase margins...
\usepackage[margin=0.75in]{geometry}
% Finite State Machine stuff

% Format tables nicely
\usepackage[latin1]{inputenc}
\usepackage{array}

\usepackage{amsfonts}
\usepackage{amssymb}
\usepackage{amsmath,amsthm}

\renewcommand{\implies}{\Rightarrow} % redefine command "implies"
\renewcommand{\iff}{\Leftrightarrow} % double arrow
\newcommand{\maps}{\rightarrow} % define command "map"
\newcommand{\union}{\cup}
\newcommand{\intersect}{\cap}
\newcommand{\N}{\mathbb{N}} % natural number
\newcommand{\Q}{\mathbb{Q}} % rational number
\newcommand{\R}{\mathbb{R}} % real number
\newcommand{\Z}{\mathbb{Z}} % integers
\newcommand\tab[1][1cm]{\hspace*{#1}} %\tab command

% Add more packages that you use here...

\begin{document}
\title{Homework 18}
\author{Brian Knotten, Brett Schreiber, Brian Falkenstein}
\maketitle

\section*{27}
A binary decision tree can be constructed which has the solution to this
problem at the leaves. Let a node $v$ at depth $i$ of the decision tree
represent a proposed solution for chapters $x_1, x_2... x_i$. Let the left
child of $v$ represent a proposed solution identical to $v$'s solution, but
considering $x_{i + 1}$ in a new volume. Let the right child of $v$ also
represent a proposed solution identical to $v$'s solution, but which appends
$x_{i + 1}$ into the same volume as $x_i$. The proposed solutions for all
chapters $x_1$ through $x_n$ will be at the leaves, however, there are $2^n$
leaves since it is a binary tree of depth $n$. So the tree must be pruned into
a polynomial algorithm, and it also must be pruned for correctness.
\\\\
The pruning rules are as follows:
\begin{enumerate}
    \item If a vertex $v$ proposes a solution with more than $k$ volumes,
        then prune $v$, since the solution cannot use more than $k$ volumes.
        \\\\
        This pruning rule does not definitively reduce the input size, but it
        makes reasoning about the tree easier.

    \item If two nodes $u$ and $v$ are on the same level, have the same
        number of volumes, and the difference between $u$'s longest and
        shortest volumes is less than or equal to the difference between $v$'s
        longest and shortest volumes, then prune $v$. $v$'s children can only
        decrease the difference between the longest and shortest volumes if
        they add to the shortest volume. And anything $v$'s children can add
        to the shortest volume, $u$ can also add. So $u$'s children will always
        have an equal or better proposed solution than $v$'s children.
        \\\\
        This pruning rule reduces the number of nodes at any given depth to $k$,
        since no two nodes at the same depth can use the same number of volumes,
        since one of them will be pruned. Therefore the tree is of size $nk$.
\end{enumerate}

With these pruning rules, the tree can be turned into a 2-dimensional array $A$,
where $A[i, j] =$ the least difference in pages between the smallest and largest
 volumes of a partition of chapters $x_1$ through $x_i$ into $j$ volumes.
\\\\
The algorithm to populate $A$ is as follows and runs in $O(nk)$:
\begin{verbatim}
for i = 1 to n do:
    A[i, 1]                   # Base case: having only 1 volume implies 0 distance

for j = 1 to k do:
    A[0, j] = 0               # Base case: 0 pages implies 0 distance


for i = 1 to n do:            # For each chapter x[i]

    for j = 1 to k do:        # For every possible number of volumes:


        A[i, j] = min(        # Take the minimum of either:
                A[i - 1, j],  # appending to the previous volume
                A[i, j]       # or of what is already in the array cell
            )

        if j < n:             # Only create a new volume if it's possible

        A[i, j + 1] = min(    # Take the minimum of either:
                A [i - 1, j], # Adding it to its own volume
                A[i, j + 1]   # or of what is already in the array cell
            )


Output A[n, k] # The solution is the minimum distance between the smallest and
               # largest volumes considering all n chapters and k volumes
\end{verbatim}

\section*{28}
\subsection*{a}
The decision tree for this problem can be constructed with the following rules:
\begin{itemize}
\item Nodes in the tree have $n$ children, where the $i'th$ child for $1<=i<=n$ corresponds to choosing $v_i$ to be a center point. Thus, the tree has height $k$ and will contain all solutions at the leaves.
\item The nodes will store the current aggregate distance, that is, the sum of all the distances from each $v\in V$ to its closest center point $c\in C$.
\end{itemize}
The total size of the tree is $n^k$. This can then be pruned with the following rules:
\begin{enumerate}
\item If a node at depth $i$ already exists in $C$, prune it. That is, if that vertex has already been added to $C$, remove that sub tree, as a valid solution contains $k$ different vertices in $C$.
\item If two nodes $u$ and $v$ at depth $i$ have the same $c_1...c_{i-1}$ center points, and $D_u < D_v$, prune $v$ (where $D_u$ is the total aggregate distance for $u$).
\end {enumerate}
The first pruning rule leaves the tree as still roughly $n^k$. However, under the second pruning rule, every depth $i$ is limited to $n$ nodes. This is because at depth $i-1$, for some node $v$, all of $v's$ children have have the same $c_1...c_{i-1}$, and only one will not be pruned, that is the one with the shortest $D$. This brings the size of the tree down to $nk$.\\
At each node $v$, a new aggregate distance must be calculated, which takes $n$ calculations, which leaves this tree with a run time of $O(n^2 k)$. \\
The dynamic algorithm is given as follows:
\begin{verbatim}
A = [k, n]                          # Construct A to be an array of size k by n

A[0, :] = infinity                  # Base case: having zero centers gives a
                                    # distance of infinity

for i = 1 to k do                   # For each row
     last_v = min(A[i-1, :])        # Retrieve the minimum distance vertex from the previous row,
                                    # and use it for calculations on the current row.

     for j = 1 to no do             # For each row, check all possible center values
         if(last_v != V[j])         # Disregard already-used vertices
               A[i, j] = new_dist(last_v, V[j]) # Calculate new distance with V[j] as a new center

output min(A[k, :])                 # Return the minimum value in the last row

\end{verbatim}
Here, A[i, j] represents the minimum aggregate distance using $i$ center points, where $v_j$ is the last vertex to be chosen as a center.
\end{document}
