\documentclass[letterpaper,notitlepage,twoside]{article}

% Basic imports, increase margins...
\usepackage[margin=0.75in]{geometry}
% Finite State Machine stuff

% Format tables nicely
\usepackage[latin1]{inputenc}
\usepackage{array}

\usepackage{amsfonts} 
\usepackage{amssymb}
\usepackage{amsmath,amsthm}

\renewcommand{\implies}{\Rightarrow} % redefine command "implies"  
\renewcommand{\iff}{\Leftrightarrow} % double arrow
\newcommand{\maps}{\rightarrow} % define command "map" 
\newcommand{\union}{\cup}
\newcommand{\intersect}{\cap}
\newcommand{\N}{\mathbb{N}} % natural number 
\newcommand{\Q}{\mathbb{Q}} % rational number 
\newcommand{\R}{\mathbb{R}} % real number 
\newcommand{\Z}{\mathbb{Z}} % integers 
\newcommand\tab[1][1cm]{\hspace*{#1}} %\tab command

% Add more packages that you use here...

\begin{document}
\title{Homework 18}
\author{Brian Knotten, Brett Schreiber, Brian Falkenstein}
\maketitle

\section*{28}
The decision tree for this problem can be constructed with the following rules:
\begin{itemize}
\item Nodes in the tree have $n$ children, where the $i'th$ child for $1<=i<=n$ corresponds to choosing $v_i$ to be a center point. Thus, the tree has height $k$ and will contain all solutions at the leaves.  
\item The nodes will store the current aggregate distance, that is, the sum of all the distances from each $v\in V$ to its closest center point $c\in C$. 
\end{itemize}
The total size of the tree is $n^k$. This can then be pruned with the following rules:
\begin{enumerate}
\item If a node at depth $i$ already exists in $C$, prune it. That is, if that vertex has already been added to $C$, remove that sub tree, as a valid solution contains $k$ different vertices in $C$. 
\item If two nodes $u$ and $v$ at depth $i$ have the same $c_1...c_{i-1}$ center points, and $D_u < D_v$, prune $v$ (where $D_u$ is the total aggregate distance for $u$). 
\end {enumerate}

\end{document}
