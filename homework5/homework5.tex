\documentclass[letterpaper,notitlepage,twoside]{article}

% Basic imports, increase margins...
\usepackage[margin=0.75in]{geometry}
\usepackage{amssymb}
\usepackage{amsmath}

% Finite State Machine stuff
\usepackage{pgf}
\usepackage{tikz}
\usetikzlibrary{arrows,automata}
\usepackage{pgfplots}

% Format tables nicely
\usepackage[latin1]{inputenc}
\usepackage{array}
\usepackage{booktabs}
\setlength{\heavyrulewidth}{1.5pt}
\setlength{\abovetopsep}{4pt}

\usepackage{amsfonts} 
\usepackage{amssymb}
\usepackage{amsmath,amsthm}

\renewcommand{\implies}{\Rightarrow} % redefine command "implies"  
\renewcommand{\iff}{\Leftrightarrow} % double arrow
\newcommand{\maps}{\rightarrow} % define command "map" 
\newcommand{\union}{\cup}
\newcommand{\intersect}{\cap}
\newcommand{\N}{\mathbb{N}} % natural number 
\newcommand{\Q}{\mathbb{Q}} % rational number 
\newcommand{\R}{\mathbb{R}} % real number 
\newcommand{\Z}{\mathbb{Z}} % integers 
\newcommand\tab[1][1cm]{\hspace*{#1}} %\tab command

% Add more packages that you use here..

\title{CS 1510 Homework 5}
\author{Brian Falkenstein, Brian Knotten, Brett Schreiber}

\begin{document}
\maketitle

\section*{11}
\subsection*{a}
$SJF$ is not optimal on the following input $I$:\\
$J_1 = (0, 2)$\\
$J_2 = (1, 2)$\\
For each time $t$:
\begin{enumerate}
\item Run $J_1$ (it is the only choice)
\item Run $J_2$ (arbitrarily, since $J_1$ and $J_2$ are the same size.)
\item Run $J_2$ (arbitrarily). $J_2$ is completed now, so $C_2 = 3$
\item Run $J_1$ (it is the only choice). $J_1$ is completed now, so $C_1 = 4$
\end{enumerate}
The total completion time for $SJF(I)$ is $C_1 + C_2 = 4 + 3 = 7$\\
But a more optimal solution $opt(I)$ exists. For each time $t$:
\begin{enumerate}
\item Run $J_1$
\item Run $J_1$. $J_1$ is now finished, so $C_1 = 2$.
\item Run $J_2$.
\item Run $J_2$. $J_2$ is now finished, so $C_2 = 4$.
\end{enumerate}
The total completion time for $opt(I)$ is $C_1 + C_2 = 2 + 4 = 6$\\
$SJF(I)$ is not optimal therefore $SJF$ is incorrect.

\subsection*{b}
This proof does not consider the possibility of job $j$ completing between times $t$ and $u$. If this were the case, and $j$ was initially scheduled to complete before time $u$, moving it back to $u$ will increase its completion time, making its output not optimal. 
\subsection*{c}
Assume that $A$, the algorithm that implements $SRPT$, is incorrect and has some input $I$ that makes it give the incorrect output. Define $Opt(I)$ to be the correct output that agrees with $A(I)$, the output from $A$ on $I$, for the most steps. Also define the first "step", or time interval that $A(I)$ and $Opt(I)$ disagree, to be $t$. At time $t$, say that $A(I)$ schedules job $J_A$ with tuple $(r_A, x_A)$, and $Opt(I)$ schedules job $J_O$ with tuple $(r_O, x_O)$. We can construct $Opt'(I)$ by simply swapping $J_O$ with the next instance of $Opt(I)$ scheduling $J_A$, say at time $u$. \\
$Opt'(I)$ clearly agrees with $A(I)$ for one more step, as it now also schedules $J_A$ at time $t$. The problem addressed in part b of this problem can also be disproven here. We know that $J_A$ has a shorter time left until completion than $J_O$, because of the definition of A. Thus, if $J_O$ were to complete between times $t$ and $u$, we could simply swap the entirety of $J_A$ into the spots that $J_O$ is ran, and have it complete even earlier, lowering the total completion time. The increased completion time added by having to shift $J_O$ down would be less than the decreased time from completing $J_A$ earlier, because $J_O$ could also be scheduled into the additional slots that $J_A$ does not need.  

\section*{17}
Consider the following algorithm $A$:\\
Given rooted tree $T$:\\
\tab Sort the leaves by value in descending order.\\
\tab For each leaf $l$ (from greatest value to least): \\ 
\tab\tab If any ancestor's current capacity is less than 0, do not include $l$ in the output
\tab\tab Otherwise, include $l$, and for each ancestor of $l$ set its new capacity to be its current capacity - 1
\end{document}
