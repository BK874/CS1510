\documentclass[letterpaper,notitlepage,twoside]{article}

% Basic imports, increase margins...
\usepackage[margin=0.75in]{geometry}
\usepackage{amssymb}
\usepackage{amsmath}

% Finite State Machine stuff
\usepackage{pgf}
\usepackage{tikz}
\usetikzlibrary{arrows,automata}
\usepackage{pgfplots}

% Format tables nicely
\usepackage[latin1]{inputenc}
\usepackage{array}
\usepackage{booktabs}
\setlength{\heavyrulewidth}{1.5pt}
\setlength{\abovetopsep}{4pt}

\usepackage{amsfonts} 
\usepackage{amssymb}
\usepackage{amsmath,amsthm}

\renewcommand{\implies}{\Rightarrow} % redefine command "implies"  
\renewcommand{\iff}{\Leftrightarrow} % double arrow
\newcommand{\maps}{\rightarrow} % define command "map" 
\newcommand{\union}{\cup}
\newcommand{\intersect}{\cap}
\newcommand{\N}{\mathbb{N}} % natural number 
\newcommand{\Q}{\mathbb{Q}} % rational number 
\newcommand{\R}{\mathbb{R}} % real number 
\newcommand{\Z}{\mathbb{Z}} % integers 
\newcommand\tab[1][1cm]{\hspace*{#1}} %\tab command

% Add more packages that you use here..

\title{CS 1510 Homework 5}
\author{Brian Falkenstein, Brian Knotten, Brett Schreiber}

\begin{document}
\maketitle

\section*{11}
\subsection*{a}
$SJF$ is not optimal on the following input $I$:\\
$J_1 = (0, 2)$\\
$J_2 = (1, 2)$\\
For each time $t$:
\begin{enumerate}
\item Run $J_1$ (it is the only choice)
\item Run $J_2$ (arbitrarily, since $J_1$ and $J_2$ are the same size.)
\item Run $J_2$ (arbitrarily). $J_2$ is completed now, so $C_2 = 3$
\item Run $J_1$ (it is the only choice). $J_1$ is completed now, so $C_1 = 4$
\end{enumerate}
The total completion time for $SJF(I)$ is $C_1 + C_2 = 4 + 3 = 7$\\
But a more optimal solution $opt(I)$ exists. For each time $t$:
\begin{enumerate}
\item Run $J_1$
\item Run $J_1$. $J_1$ is now finished, so $C_1 = 2$.
\item Run $J_2$.
\item Run $J_2$. $J_2$ is now finished, so $C_2 = 4$.
\end{enumerate}
The total completion time for $opt(I)$ is $C_1 + C_2 = 2 + 4 = 6$\\
$SJF(I)$ is not optimal therefore $SJF$ is incorrect.

\section*{17}
Consider the following algorithm $A$:\\
Given rooted tree $T$:\\
\tab Sort the leaves by value in descending order.\\
\tab For each leaf $l$ (from greatest value to least): \\ 
\tab\tab If any ancestor's current capacity is less than 0, do not include $l$ in the output \\
\tab\tab Otherwise, include $l$, and for each ancestor of $l$ set its new capacity to be its current capacity - 1 \\

Proof: Assume to reach a contradiction that our algorithm, hereafter referred to as $A$, is incorrect. Then there exists an input $I$ on which $A$ produces a suboptimal output. Let $OPT(I)$ be the optimal solution to the problem that agrees with $A(I)$ for the max number of steps i.e. up to leaf $n$ $OPT(I)$ and $A(I)$ have included and excluded the same leaves. Because each step is either including or excluding a leaf, the disagreement can be one of two cases:\\
\begin{enumerate}
  \item $A(I)$ excluded leaf $n$ and $OPT(I)$ included leaf $n$:\\
  Because $A(I)$ and $OPT(I)$ agreed on every leaf up to this step and $A$ always considers the next highest possible value and only excludes leaves when a leaf's ancestors' capacity does not accommodate it, it cannot be the case that $OPT(I)$ includes $n$ and $A(I)$ does not. 

  \item $A(I)$ included leaf $n$ and $OPT(I)$ excluded leaf $n$:\\
    $OPT(I)$ excludes a leaf that $A(I)$ includes either because the parent nodes are at capacity or because there is a better leaf it will select later on. The first case cannot occur, as $OPT(I)$ and $A(I)$ have the same capacity because they have agreed on every leaf up to this point. In the second case, because $A$ always selects the next highest possible value and $OPT(I)$ has agreed with $A(I)$ up to this point, it is impossible for there to be a leaf with greater value than $n$ available for $OPT(I)$ to select later on. Therefore the leaf $OPT(I)$ would select instead of $n$ has to be of equal or lesser value than $n$ and $OPT(I)$'s solution is equivalent to or lesser than $A(I)$'s solution.\\
\end{enumerate}
Therefore $OPT(I)$ can be modified into $OPT'(I)$ where $OPT'(I)$ agrees with $A(I)$ for one more step (including or excluding $n$), contradicting the statement that $OPT(I)$ is the optimal solution that agrees with $A(I)$ for the most number of steps.\\
Thus, by contradiction, $A(I)$ is correct.
\end{document} \\
