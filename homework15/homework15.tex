\documentclass[letterpaper,notitlepage,twoside]{article}

% Basic imports, increase margins...
\usepackage[margin=0.75in]{geometry}
% Finite State Machine stuff

% Format tables nicely
\usepackage[latin1]{inputenc}
\usepackage{array}

\usepackage{amsfonts} 
\usepackage{amssymb}
\usepackage{amsmath,amsthm}

\renewcommand{\implies}{\Rightarrow} % redefine command "implies"  
\renewcommand{\iff}{\Leftrightarrow} % double arrow
\newcommand{\maps}{\rightarrow} % define command "map" 
\newcommand{\union}{\cup}
\newcommand{\intersect}{\cap}
\newcommand{\N}{\mathbb{N}} % natural number 
\newcommand{\Q}{\mathbb{Q}} % rational number 
\newcommand{\R}{\mathbb{R}} % real number 
\newcommand{\Z}{\mathbb{Z}} % integers 
\newcommand\tab[1][1cm]{\hspace*{#1}} %\tab command

% Add more packages that you use here...

\begin{document}
\title{Homework 15}
\author{Brian Knotten, Brett Schreiber, Brian Falkenstein}
\maketitle

\section*{23}
This problem is a variation of the set partition problem where we are partitioning a set of size $n$ (the $n$ request times) into $k$ subsets (each subset is the collection of requests satisfied if the information is sent at one of the $k$ broadcasts). There are a max of ${n-1 \choose k-1}$ possible partitions: there are $n$ possible times for each of the $k$ broadcasts (time 2 to time $n+1$), but the final broadcast must go after the last nonzero request time, so there are a max of $n-1$ possible times for each of the $k-1$ broadcasts.\\
Let $B$ be the list of $k$ broadcasts and let $m$ be the time of the last nonzero request.The algorithm starts by placing each of the $k$ broadcasts at the farthest feasible time i.e.: \\
for broadcast $i = 1$ to $k$:\\
\tab place partition i at slot m+2-i\\
so that the first (assigned) broadcast is at time $m+1$, the second broadcast is at time $m$, etc. \\
Then: let $minSum = \infty$ and let $bTimes =$ the current positions of the $k$ broadcasts. \\
for broadcasts $i = k$ to 1:\\
\tab let $\ell$ be the number of broadcasts sent thus far (i.e. if $i = k$, one sent if $i=k-1$, two sent, etc.)\\
\tab for each ${n-i+1 \choose \ell}$ possible broadcast time:\\
\tab\tab if $totalWaitTime \leq minSum$:\\
\tab\tab\tab $minSum = totalWaitTime$\\
\tab\tab\tab $bTimes =$ current positions of the $k$ broadcasts\\
The first for loop iterates over the $k$ broadcasts. The second for loop iterates over the $k$ broadcasts, and for each loop it considers a max of ${n \choose k}$ combinations. Note that due to the symmetry of binomial coefficients, ${n \choose k} = {n \choose n-k}$. Therefore ${n \choose k}$ reaches is max value when $k = \lfloor\dfrac{n}{2}\rfloor$ or $\lceil\dfrac{n}{2}\rceil$. Therefore the algorithm is $O(min(n^{k}, n^{n-k}))$

\section*{23}
This problem is a variation of the set partition problem where we are partitioning a set of size $n$ (the $n$ request times) into $k$ subsets (each subset is the collection of requests satisfied if the information is sent at one of the $k$ broadcasts). The decision tree for this problem can be thought of as:
\begin{itemize}
\item Each node has a branching factor of $n$ and stores the cumulative waiting times given the partitions applied up until that point.
\item The children of each node represent placing a partition at that position in $R$. So, the first child represents placing a partition at $R_0$, the second child a partition at $R_1$, etc. 
\item At depth $i$, $i$ partitions have been placed. 
\end{itemize}
Thus, the root of the tree would be null, and the first level would contain the waiting times resulting from placing a partition at $R_0...R_n$. Note that the sum of the waiting times in solutions that do not broadcast pages to users (that is, there are users requesting the page after the last partition) is $\infty$, because a valid solution must have every request met.\\
This will cover all $\binom{n}{k}$ possible ways to partition $R$ $k$ times. This tree can then be pruned and turned into a dynamic programming algorithm with the following rules:
\begin{enumerate}
\item If a node has more than $k$ partitions, prune it. (this limits the height of the tree to $k$).
\item For nodes at the same depth (IE have made the same number of partitions), prune all but the one with the shortest cumulative wait time. 
\end{enumerate}
This limits the height of the tree to $k$, and the number of nodes at each depth to $n$ (the sums must be calculated, and then pruned). The solution would then be found at the leaves, and would be the minimum sum over all the leaves. \\
This tree can then be imagined as an $n\times k$ table, where the rows represent the number of partitions (the depth of the tree) and the columns represent the places you can place the next partition (the children of each node in the tree). $A[i, j] =$ The total waiting time of adding the $j$th partition at time $i$. \\
A supplemental array $Broadcasts[n]$ is needed, where $Broadcasts[i] = 1$ if the optimal solution should broadcast at time $i + 1$, otherwise, $Broadcasts[i] = 0$.
This algorithm is polynomial, as it takes time $nk$ to traverse the table, and $n$ time to compute the sum at each index, resulting in a runtime of $O(n^3)$.
\begin{verbatim}
for i = 1 to n do:
    Broadcasts[i] = 0

for i = n to 1 do:                # Initialize first row. Set all solutions that do not meet all 
   if R[i-1] != 0:                # requests to infinity. So the first broadcast should always
       R[i] = waitTime(i)         # be immediately after the last request.
       Broadcasts[i] = 1
       for j = 1 to i do:
           R[j] = infinity         

    for j = 2 to k do:                            # For each new opportunity to broadcast (level of the tree)
        minimum_time = n                          # Keep track of the broadcast time which minimizes cost
        
        for i = n to 1 do:                        # For each time that can broadcast (node in the tree level)        
            sum = 0                               # Calculate the wait time with the current broadcasts
            time_waiting = 1
            for s = n to 1 do:                    # For each time interval
                if Broadcasts[s] == 1 or s == i:  # If this time has a previous or currently-considered broadcast,
                    time_waiting = 1              # Reset the wait-time
                
                sum = sum + R[s] * time_waiting   # Sum up the wait times of each requester
                multiplier = multiplier + 1
                
            A[i, j] = sum
            
            if A[i, j] < A[i, minimum_time]:      # If a smaller wait time is found, redefine the minimum time
                minimum_time = j
        
        Broadcasts[minimum_time] = 1              # After all wait times have been considered,
                                                  # broadcast at the time which minimizes wait time

Output Broadcasts                                 # Output the whole binary Broadcasts array,
                                                  # which gives the solution of when to broadcast.
\end{verbatim}

\section*{26}

The decision tree in this problem is similar to that in the previous problem, but instead of having a branching factor of just $n$, it will have a branching factor of $n\times j$, in order to account for placing the partition at any one of the $n$ times for all $j$ pages. The table will also be similar, but will now need a 3rd dimension to account for each page. That is, $A[p, q, r]$ represents: the minimum total waiting time of broadcasting Page $r$ at time $p + 1$, having made $q$ broadcasts in total. \\
The broadcast array will also get another dimension. Now, $Broadcast[p, r] = 1$ if page $r$ is broadcast at time $p + 1$, otherwise, $Broadcast[p, r] = 0$. \\

\begin{verbatim}

\end{verbatim}

\end{document}

