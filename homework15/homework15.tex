\documentclass[letterpaper,notitlepage,twoside]{article}

% Basic imports, increase margins...
\usepackage[margin=0.75in]{geometry}
% Finite State Machine stuff

% Format tables nicely
\usepackage[latin1]{inputenc}
\usepackage{array}

\usepackage{amsfonts} 
\usepackage{amssymb}
\usepackage{amsmath,amsthm}

\renewcommand{\implies}{\Rightarrow} % redefine command "implies"  
\renewcommand{\iff}{\Leftrightarrow} % double arrow
\newcommand{\maps}{\rightarrow} % define command "map" 
\newcommand{\union}{\cup}
\newcommand{\intersect}{\cap}
\newcommand{\N}{\mathbb{N}} % natural number 
\newcommand{\Q}{\mathbb{Q}} % rational number 
\newcommand{\R}{\mathbb{R}} % real number 
\newcommand{\Z}{\mathbb{Z}} % integers 
\newcommand\tab[1][1cm]{\hspace*{#1}} %\tab command

% Add more packages that you use here...

\begin{document}
\title{Homework 15}
\author{Brian Knotten, Brett Schreiber, Brian Falkenstein}
\maketitle

\section*{23}
This problem is a variation of the set partition problem where we are partitioning a set of size $n$ (the $n$ request times) into $k$ subsets (each subset is the collection of requests satisfied if the information is sent at one of the $k$ broadcasts). There are a max of ${n-1 \choose k-1}$ possible partitions: there are $n$ possible times for each of the $k$ broadcasts (time 2 to time $n+1$), but the final broadcast must go after the last nonzero request time, so there are a max of $n-1$ possible times for each of the $k-1$ broadcasts.\\
Let $B$ be the list of $k$ broadcasts and let $m$ be the time of the last nonzero request.The algorithm starts by placing each of the $k$ broadcasts at the farthest feasible time i.e.: \\
for broadcast $i = 1$ to $k$:\\
\tab place partition i at slot m+2-i\\
so that the first (assigned) broadcast is at time $m+1$, the second broadcast is at time $m$, etc. \\
Then: let $minSum = \infty$ and let $bTimes =$ the current positions of the $k$ broadcasts. \\
for broadcasts $i = k$ to 1:\\
\tab let $\ell$ be the number of broadcasts sent thus far (i.e. if $i = k$, one sent if $i=k-1$, two sent, etc.)\\
\tab for each ${n-i+1 \choose \ell}$ possible broadcast time:\\
\tab\tab if $totalWaitTime \leq minSum$:\\
\tab\tab\tab $minSum = totalWaitTime$\\
\tab\tab\tab $bTimes =$ current positions of the $k$ broadcasts\\
The first for loop iterates over the $k$ broadcasts. The second for loop iterates over the $k$ broadcasts, and for each loop it considers a max of ${n \choose k}$ combinations. Note that due to the symmetry of binomial coefficients, ${n \choose k} = {n \choose n-k}$. Therefore ${n \choose k}$ reaches is max value when $k = \lfloor\dfrac{n}{2}\rfloor$ or $\lceil\dfrac{n}{2}\rceil$. Therefore the algorithm is $O(min(n^{k}, n^{n-k}))$
\section*{26}

\end{document}

