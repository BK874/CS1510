\documentclass[letterpaper,notitlepage,twoside]{article}

% Basic imports, increase margins...
\usepackage[margin=0.75in]{geometry}
% Finite State Machine stuff

% Format tables nicely
\usepackage[latin1]{inputenc}
\usepackage{array}

\usepackage{amsfonts} 
\usepackage{amssymb}
\usepackage{amsmath,amsthm}

\renewcommand{\implies}{\Rightarrow} % redefine command "implies"  
\renewcommand{\iff}{\Leftrightarrow} % double arrow
\newcommand{\maps}{\rightarrow} % define command "map" 
\newcommand{\union}{\cup}
\newcommand{\intersect}{\cap}
\newcommand{\N}{\mathbb{N}} % natural number 
\newcommand{\Q}{\mathbb{Q}} % rational number 
\newcommand{\R}{\mathbb{R}} % real number 
\newcommand{\Z}{\mathbb{Z}} % integers 
\newcommand\tab[1][1cm]{\hspace*{#1}} %\tab command

% Add more packages that you use here...

\begin{document}
\title{Homework 14}
\author{Brian Knotten, Brett Schreiber, Brian Falkenstein}
\maketitle

\section*{19}
\subsection*{a}
A recursive algorithm for this problem is as follows:
In short, keep track of the current indices of $T$ and $P$ to find the optimal solution of either taking one from $T$ to match with $P$ or not. After exhausting all possible options to take, return the choices that yield the maximum cost. 
\begin{verbatim}
T[n]
C[n]
P[k]

A(i, j):
    if j == 0 or i == 0: return 0 # No more opportunities

    if T[i] == P[j]:
        return max(
            A(i - 1, j - 1) + C[i], # Either take the opportunity and use T[i] for P[j], therefore moving to the next P to fill...
            A(i - 1, j) # Or, skip this T[i], and look for a better one to fill P[j]
        )
		
    else:
        return A(i - 1, j)
	

Output A(n, k)
\end{verbatim}

To convert this into a dynamic programming algorithm, use a 2-dimensional array to keep track of the indices $i$ and $j$ where $A[i][j]$ means: The maximum cost achievable considering the first $i$ items of $T$ and the first $j$ items of $P$. \\

Filling this array $A$ requires time $O(nk)$, where $n$ is the number of items in $T$ and $k$ is the number of items in $P$. Since $k \leq n$, the time complexity is therefore $O(n^2)$ in the worst case. \\

Here is the code for the dynamic solution:
\begin{verbatim}
# Base case: with none of T or P to consider, the only available cost is 0.
A[0][0] = 0

for i = 1 to n do:
    for j = 1 to k do:
        
	if T[i] == P[j]:
	    A[i][j] = max(
                A[i - 1][j - 1] + C[i],
                A[i - 1][j]
            )
        else:
            A[i][j] = A[i - 1][j]


Output A[n][k]
\end{verbatim}

\subsection*{b}
You can alternatively enumerate all the subsequences of T as a binary tree, where, given a vertex at depth $i$, the left child represents omitting $T[i + 1]$ and the right child represents appending $T[i + 1]$ with cost $C[i + 1]$ into the solution set. Since it is a binary tree, it has $2^n$ leaves representing solutions, so the following pruning rules must be applied.
\begin{enumerate}
\item For any node $v$, if the sequence in $v$ is not a prefix of $P$, (for example, $v = XYY$ and $P = XYXX$, then prune $v$. This also applies to any $v$ larger than $P$, since a string $a$ cannot be a prefix of $b$ if $|a| > |b|$. $v$ cannot possibly be a solution since the children of $v$ only append to the end of $v$'s sequence, and cannot therefore rectify any differences between $v$'s sequence and $P$. This limits the depth of the tree from $n$ to $k$, since any sequence larger than $k$ cannot be a prefix of $P$ since $|P| = k$.
\item For any two nodes $u$ and $v$ at the same level where the string $u = $ string $v$, if the cost of $u$ is greater than or equal to the cost of $v$, then prune the tree rooted at $v$. Any item $v$ can append in the future, $u$ can also append and still have a greater or equal total cost. So $v$'s tree does not contain the solution.
\end{enumerate}
Applying both pruning rules leaves at most $i$ nodes at depth $i$, rather than $2^i$ nodes, since there can only be one node representing a substring of size $j$ where $0 \leq j \leq i$, since there is only one valid prefix of length $j$ and only one prefix that has the highest cost. So there are $i$ nodes at depth $i$: the maximum cost prefix of length $0$, the maxmimum cost prefix of length $1$, ... the maximum cost prefix of length $i - 1$, and the maximum cost prefix of length $i$. This is $i$ strings in total at depth $i$. \\

Since the tree is of depth $k$ and breadth at most $n$, we can make a polynomial algorithm using a 2d array $A[k][n]$, where $A[i][j]$ represents the maximum aggregate cost of the prefix of the first $i$ elements of $P$ of length $j$. Here is the dynamic algorithm:
\begin{verbatim}

for i = 1 to k do: # For each level of the tree
    for j = 1 to i do: # For each node at depth i (containing a prefix of size j)
        if T[i + 1] == P[i + 1]: # Only consider adding a new element if it matches the prefix
            A[i + 1][j + 1] = max(
                A[i + 1][j + 1] # Compete with what's currently in the cell
                A[i][j] + C[i + 1]
            )

Output A[k][n]
\end{verbatim}

\subsection*{c}
Consider a tree of height $k$ where each node has $n$ children. At depth $i$, we are considering the $j$th child where $1 \leq j \leq n$  such that the child represents appending $T[j]$ to the solution, (that is, matching $P[i + 1]$ with $T[j]$). The solution would be at the leaves, where a certain node will have matched $P$ from $0$ to $k$. But the tree has a branching factor of $n$, so it needs to be pruned down from $n^k$ leaves. Here are the pruning rules:
\begin{enumerate}
\item Given a node $v$ at depth $i$, prune $v$ if the last character in the sequence of $v$ is not equal to $P[i]$. This means that the matching is not correct, and it cannot be corrected by adding more onto it. So the solution will not be in the tree rooted at $v$.
\item Given a node $u$ at depth $i$, if $u$ is the $j$th child of his parent node $v$, prune the $0..j$th child of $u$. These children are adding characters which are not in increasing order, so they cannot be a part of the solution. So prune these children of $u$.
\end{enumerate}

\section*{22}
Pruning rule:
\begin{enumerate}
\item If two nodes at the same depth have purchased the same amount of cards, prune the one that has the higher cost. 
\end{enumerate}
This pruning rule alone is enough to bring the size of the tree down to $n\times n$. The height of the tree will be $n$, as at each level $i$, we decide whether to purchase a card at $d_i$, and there are $n$ dates in $D$. Further, at each level $i$, there will be at max $i$ purchased cards, and we will only have one node corresponding to each possible number of cards purchased.  \\
Thus, the tree can be constructed with a table of size $n\times n$.

$A[i][j]$ = the lowest cost of the first $j$ trips having had the opportunity to purchase $i$ Bahncards. Output the minimum value over the last column (considered all n trips). 

\begin{verbatim} 
A[0][0] = 0 # With no tickets and no trips, the total cost is 0

# Fill in the first row
for j = 1 to n do:
    A[0][i] = f[i] + A[0][j - 1] # The sum of all the trips (without any discounts) is the sum of the previous trips plus the fare of the current trip.
    
for j = 1 to n do: # Going down each row of the tree for each trip j
    for i = j to n do: # Choosing to purchase i tickets at each level
        A[i + 1][j + 1] = A[i][j] + f[i] + min(
	    0, # Not purchasing a ticket
	    A[i][j] + B + f[i] - (sum of the costs of trips from d[j] to d[j] + L) / 2 # Choosing to buy a ticket for this trip
	)


Output A[n][n] # Output the cost of taking n trips, having had the opportunity to purchase n Bahncards
\end{verbatim}

The result will be in the bottom right, since it will have had n opportunities to purchase Bahncards, but the algorithm can choose not to purchase the $i$th Bahncard. To retrieve the dates of purchase, keep track of the dates when a Bahncard is worth it to purchase (this can be tracked in a boolean 1-dimensional array where each index $i$ indicates purchasing a ticket at date $d_i$).

\end{document}
