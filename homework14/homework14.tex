\documentclass[letterpaper,notitlepage,twoside]{article}

% Basic imports, increase margins...
\usepackage[margin=0.75in]{geometry}
% Finite State Machine stuff

% Format tables nicely
\usepackage[latin1]{inputenc}
\usepackage{array}

\usepackage{amsfonts} 
\usepackage{amssymb}
\usepackage{amsmath,amsthm}

\renewcommand{\implies}{\Rightarrow} % redefine command "implies"  
\renewcommand{\iff}{\Leftrightarrow} % double arrow
\newcommand{\maps}{\rightarrow} % define command "map" 
\newcommand{\union}{\cup}
\newcommand{\intersect}{\cap}
\newcommand{\N}{\mathbb{N}} % natural number 
\newcommand{\Q}{\mathbb{Q}} % rational number 
\newcommand{\R}{\mathbb{R}} % real number 
\newcommand{\Z}{\mathbb{Z}} % integers 
\newcommand\tab[1][1cm]{\hspace*{#1}} %\tab command

% Add more packages that you use here...

\begin{document}
\title{Homework 14}
\author{Brian Knotten, Brett Schreiber, Brian Falkenstein}
\maketitle

\section*{19}
\subsection*{a}
A recursive algorithm for this problem is as follows:
In short, keep track of the current indices of $T$ and $P$ to find the optimal solution of either taking one from $T$ to match with $P$ or not. After exhausting all possible options to take, return the choices that yield the maximum cost. 
\begin{verbatim}
T[n]
C[n]
P[k]

A(i, j):
    if j == 0 or i == 0: return 0 # No more opportunities

    if T[i] == P[j]:
        return max(
            A(i - i, j - i) + C[i], # Either take the opportunity and use T[i] for P[j], therefore moving to the next P to fill...
            A(i - 1, j) # Or, skip this T[i], and look for a better one to fill P[j]
        )
		
    else:
        return A(i - 1, j)
	

Output A(n, k)
\end{verbatim}

To convert this into a dynamic programming algorithm, use a 2-dimensional array to keep track of the indices $i$ and $j$ where $A[i][j]$ means: The maximum cost achievable considering the first $i$ items of $T$ and the first $j$ items of $P$. \\

Filling this array $A$ requires time $O(nk)$, where $n$ is the number of items in $T$ and $k$ is the number of items in $P$. Since $k \leq n$, the time complexity is therefore $O(n^2)$ in the worst case. \\

Here is the code for the dynamic solution:
\begin{verbatim}
# Base case: with none of T or P to consider, the only available cost is 0.
A[0][0] = 0

for i = 1 to n do:
    for j = 1 to k do:
        
	if T[i] == P[j]:
	    A[i][j] = max(
                A[i + 1][j + 1] + C[i],
                A[i + 1][j]
            )
        else:
            A[i][j] = A[i + 1][j]


Output A[n][k]
\end{verbatim}

\subsection*{b}
You can alternatively enumerate all the subsequences of T as a binary tree, where, given a vertex at depth $i$, the left child represents omitting $T[i + 1]$ and the right child represents appending $T[i + 1]$ with cost $C[i + 1]$ into the solution set. Since it is a binary tree, it has $2^n$ leaves representing solutions, so the following pruning rules must be applied.
\begin{enumerate}
\item For any node $v$, if the sequence in $v$ is not a prefix of $P$, (for example, $v = XYY$ and $P = XYXX$, then prune $v$. This also applies to any $v$ larger than $P$, since a string $a$ cannot be a prefix of $b$ if $|a| > |b|$. $v$ cannot possibly be a solution since the children of $v$ only append to the end of $v$'s sequence, and cannot therefore rectify any differences between $v$'s sequence and $P$.
\item For any two nodes $u$ and $v$ at the same level that represent the same prefix of $P$, if the cost of $u$ is greater than the cost of $v$, then prune the tree rooted at $v$. Any item $v$ can append in the future, $u$ can also append and still have a greater total cost. So $v$'s tree does not contain the solution.
\end{enumerate}

\section*{22}

\end{document}
