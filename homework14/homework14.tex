\documentclass[letterpaper,notitlepage,twoside]{article}

% Basic imports, increase margins...
\usepackage[margin=0.75in]{geometry}
% Finite State Machine stuff

% Format tables nicely
\usepackage[latin1]{inputenc}
\usepackage{array}

\usepackage{amsfonts} 
\usepackage{amssymb}
\usepackage{amsmath,amsthm}

\renewcommand{\implies}{\Rightarrow} % redefine command "implies"  
\renewcommand{\iff}{\Leftrightarrow} % double arrow
\newcommand{\maps}{\rightarrow} % define command "map" 
\newcommand{\union}{\cup}
\newcommand{\intersect}{\cap}
\newcommand{\N}{\mathbb{N}} % natural number 
\newcommand{\Q}{\mathbb{Q}} % rational number 
\newcommand{\R}{\mathbb{R}} % real number 
\newcommand{\Z}{\mathbb{Z}} % integers 
\newcommand\tab[1][1cm]{\hspace*{#1}} %\tab command

% Add more packages that you use here...

\begin{document}
\title{Homework 14}
\author{Brian Knotten, Brett Schreiber, Brian Falkenstein}
\maketitle

\section*{19}
\subsection*{a}
\begin{verbatim}
T[n]
C[n]
P[k]

A(i, j):
	  if j > k or i > n: return 0 # No more opportunities

	  if t[i] == p[j]:
		  return max(
		  	A(i + i, j + i) + C[i], # Either take the opportunity and use T[i] for P[j], therefore moving to the next P to fill...
		  	A(i + 1, j) # Or, skip this T[i], and look for a better one to fill P[j]
		  )
		
	  else:
		  return A(i + 1, j)
\end{verbatim}

\section*{22}

\end{document}
