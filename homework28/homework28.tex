
\documentclass[letterpaper,notitlepage,twoside]{article}

% Basic imports, increase margins...
\usepackage[margin=0.75in]{geometry}
% Finite State Machine stuff
\usepackage{pgf}
\usepackage{tikz}
\usetikzlibrary{arrows, automata}
% Format tables nicely
\usepackage[latin1]{inputenc}
\usepackage{array}

\usepackage{amsfonts}
\usepackage{amssymb}
\usepackage{amsmath,amsthm}

\renewcommand{\implies}{\Rightarrow} % redefine command "implies"
\renewcommand{\iff}{\Leftrightarrow} % double arrow
\newcommand{\maps}{\rightarrow} % define command "map"
\newcommand{\union}{\cup}
\newcommand{\intersect}{\cap}
\newcommand{\N}{\mathbb{N}} % natural number
\newcommand{\Q}{\mathbb{Q}} % rational number
\newcommand{\R}{\mathbb{R}} % real number
\newcommand{\Z}{\mathbb{Z}} % integers
\newcommand\tab[1][1cm]{\hspace*{#1}} %\tab command

% Add more packages that you use here...

\begin{document}
\title{Homework 28}
\author{Brian Knotten, Brett Schreiber, Brian Falkenstein}
\maketitle
\section*{7}
Calculating the result of plugging $k$ into a polynomial of degree $n$with coefficients $c_1...c_n$ can be done in $\log n$ using $\frac{n}{\log n}$ processors and the following algorithm:
\\\\
First, make $\frac{n}{\log n}$ copies of $k$. $k$ is only stored in one memory location, but later in the algorithm, many processors will need to access its value. So first the algorithm will spend $\log n$ time to make $\frac{n}{\log n}$ copies. Let $k$ be originally stored in memory location $l_1$. Use one processor to read $k$ from $l_1$ and copy $k$ into $l_2$. Now use two processors: one to read $k$ from $l_1$ and copy into $l_3$, and one to read $k$ from $l_2$ and copy into $l_4$. Every constant time interval, the number of memory locations that hold $k$ can double. At the $\log \frac{n}{\log n}$th time interval, $\frac{n}{2 \log n}$ processors double $\frac{n}{2 \log n}$ copies of $k$ to get $\frac{n}{\log n}$ total copies. So copying $k$ for each processor takes $\log \frac{n}{\log n}$ steps.
\\
Next, divide the coefficient list $c_1...c_n$ into $\frac{n}{\log n}$ equal slice. Since there are $n$ coefficients, there are $\log{n}$ coefficients in each equal slice. Assign every processor to a slice of the coefficient list. There are as many processors as there are partitions of the coefficient list.
\\
Each processor can perform the following task independently and at the same time: Given a unique copy of $k$ and the coefficient list slice $c_i, c_{i + 1}, c_{i + 2}, ... c_{i + \frac{n}{\log n}}$, compute the number $c_ik^i + c_{i + 1}k^{i + 1} + c_{i + 2}k^{i + 2} + ... c_{i + \frac{n}{\log n}}k^{i + \frac{n}{\log n}}$. Each processor will take $\log n$ steps to compute this since adding and multiplying takes constant time, and there are $\log n$ coefficients in each slice of the coefficient list.
\\
At this point there are $\frac{n}{\log n}$ numbers which were computed above. To form the full polynomial result these numbers have to be summed up. $\frac{n}{2 \log n}$ processors can each independently add the pairs of numbers from each adjacent coefficient slice. This will give a result of $\frac{n}{2 \log n}$ numbers which $\frac{n}{2 \log n}$ processors can then add. Every constant time interval, the number of summed numbers halves. At the $\log \frac{n}{\log n}$th time interval, there will be exactly one number left, which is the result of plugging $k$ into the polynomial with coefficients $c_1...c_n$.
\\\\
This algorithm takes $O(\log n)$ steps, since making $\frac{n}{\log n}$ copies of $k$ takes $\log \frac{n}{\log n}$ steps (less than $\log n$), the coefficient slice computation takes $\log n$ steps, and the summation of the $\frac{n}{\log n}$ slice results takes $\log \frac{n}{\log n}$ steps. This is an EREW algorithm, since $k$ is copied to ensure that no two processors are reading from the same $k$, and each processor's results are stored independently and summed into one result.
\\
The efficiency of this algorithm, assuming $S(n) = n$, is $E(n, p = \frac{n}{\log n}) = \frac{n}{p * \log n} = \frac{n}{\frac{n}{\log n} \log n} = \frac{n}{n} = 1$.

\end{document}
